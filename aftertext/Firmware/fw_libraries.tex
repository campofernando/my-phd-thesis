\chapter{Bibliotecas de \textit{firmware} CLEAN}\label{appendix:firmware-libraries}

O \textit{firmware} dos dispositivos foi desenvolvido no \textit{Framework Arduino}, que é uma abstração de códigos-fonte e bibliotecas comuns a diversas plataformas de \textit{hardware}. Esta estrutura torna possível escrever programas para controlar uma ampla gama de placas microcontroladoras de Arduino e de outros fabricantes. O \textit{framework} fornece bibliotecas de código escritas em C/C++ para programação de microcontroladores e interação com dispositivos periféricos.

Para programar todas as funcionalidades do \textit{firmware} CLEAN, o código foi estruturado em um conjunto de classes em C++. Esta estrutura foi concebida visando sua reutilização em outras plataformas de microcontroladores e outros componentes de hardware suportados no Framework Arduino (como ESP8266 da Espressif) e também para facilitar a revisão e manutenção do código. As classes desenvolvidas para o projeto estão distribuídas em quatro módulos principais, conforme mostrado na Figura 1: o módulo \textit{Hardware Interfaces}, o módulo \textit{System Drivers}, o módulo \textit{}{Sensors} e o módulo \textit{Data}.



O módulo \textit{Hardware Interfaces} agrupa todas as classes e estruturas utilizadas para se comunicar com o \textit{hardware} periférico, como sensores, módulos de temporização, módulos de geolocalização e módulos de armazenamento. Os \textit{Drivers} implementam funcionalidades que podem ser utilizadas pelo programa principal independentemente do \textit{hardware} utilizado em cada dispositivo. O pacote \textit{Sensors} está no mesmo nível dos \textit{Drivers} e pode ser interpretado como um conjunto de drivers especiais para os sensores, mas com a particularidade de ser específico para cada fabricante. Por fim, o pacote Dados engloba todas as funcionalidades relacionadas à preparação de dados de sensores para armazenamento e transmissão. Este pacote abstrai as informações de concentração adquiridas pelos sensores de gás a partir de detalhes específicos sobre o funcionamento e operação de seu hardware.

