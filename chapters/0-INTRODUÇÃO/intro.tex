% ----------------------------------------------------------
\chapter{Introdução}
% ----------------------------------------------------------

A poluição do ar é de grande risco para a saúde, sendo responsável pela morte de cerca de sete milhões de pessoas em todo o mundo anualmente e de variadas doenças cardiorrespiratórias \cite{who2021}. Inclusive tem sido evidenciado que o risco de sofrer doenças, ou mesmo de morte, é maior nos grupos de população mais vulneráveis e marginalizadas, como as minorias raciais/étnicas e as pessoas de baixo nível socioeconômico \cite{Jbaily2022AirGroups}. Embora quase 99\% da população mundial respire ar que excede os limites das diretrizes da OMS \cite{who2021}, a cobertura de monitoramento da qualidade do ar ainda é baixa e insuficiente \cite{Munir2019AnalysingSheffield}, especialmente em países de baixo e médio rendimento. Um monitoramento representativo da poluição atmosférica é essencial para a gestão eficaz da qualidade do ar, já que a variabilidade espaço-temporal da concentração de poluentes é elevada, especialmente em ambientes urbanos \cite{Kumar2015,Mead2013TheNetworks}.

Os poluentes atmosféricos têm sido monitorados usando equipamentos complexos e caros em locais fixos \cite{Kang2022PerformanceReview}. Os altos custos desses instrumentos limitam sua implantação a apenas algumas estações por cidade, deixando grandes áreas geográficas descobertas \cite{Munir2019AnalysingSheffield}, restringindo a resolução espacial e a distribuição das redes de monitoramento \cite{Jiao2016CommunityStates,Kumar2015}. No Brasil, por exemplo, as redes de monitoramento instaladas cobrem apenas aproximadamente 2\% do total de municípios brasileiras, e seu desempenho a longo prazo é muitas vezes comprometido pela falta de manutenção e de pessoal qualificado \cite{Vormittag2021AnaliseBrasil}.

Buscando uma melhor compreensão do processo de poluição do ar e seus impactos, a ideia de sensoriamento em alta resolução espacial atraiu a atenção da comunidade de qualidade do ar e o uso de monitores de baixo custo tem ganhado popularidade \cite{Motlagh2020TowardMonitoring,Kumar2015}. Devido à sua versatilidade e custos baixos (tanto de aquisição quanto de operação), esses dispositivos podem complementar a escassez espacial e temporal de redes certificadas de qualidade do ar e expandir o horizonte de novas aplicações de monitoramento \cite{Lewis2016EvaluatingResearch}. Os monitores de baixo custo, além de seu baixo preço, caracterizam-se por suas pequenas dimensões e o consumo energético reduzido \cite{Lewis2018Low-costApplications}. No entanto, esses dispositivos ainda devem alcançar níveis de confiabilidade apropriados para serem utilizados de forma estendida \cite{Penza2020Low-costMonitoring}. A literatura reporta diversos trabalhos que buscam reduzir o erro e a incerteza das medições dos sensores de baixo custo mediante a aplicação de modelos de calibração e de compensação \cite{Maag2018ADeployments, Concas2021LOW-COSTPREPRINT}. Dentre os modelos de calibração aplicados, as regressões multivariadas paramétricas e não paramétricas têm produzido os melhores resultados \cite{Feng2019ReviewTechnology,Concas2021LOW-COSTPREPRINT}.

Aqui, apresentamos a Rede Colaborativa de Qualidade do Ar Ambiental de Baixo Custo (CLEAN). A iniciativa inclui soluções de hardware (com documentação e tutoriais), bibliotecas de firmware bem documentadas e um aplicativo Web para visualização e análise de dados em tempo real. CLEAN facilita a incorporação de novos periféricos e sensores de hardware, independentemente dos requisitos da API, garantindo a interação com a aplicação Web independente do hardware desenvolvido para aplicações específicas de monitoramento.

CLEAN permite a colaboração de outros grupos e indivíduos interessados no desenvolvimento de dispositivos de monitoramento de baixo custo e dados abertos para análises ambientais. A plataforma web Renovar fornece uma API que permite que diversos dispositivos de monitoramento de baixo custo enviem seus dados para um servidor remoto para visualização e armazenamento em tempo real e geo-localizados. Além disso, a API possibilita a integração com outras aplicações Web para visualização e análise de dados. Esses dados permanecem abertamente disponíveis para posterior processamento e análise. Dada a grande versatilidade dos sensores de baixo custo, muitas aplicações poderiam ser monitoradas a partir de diversos cenários contribuindo para uma maior disponibilidade de volumes de dados.

% ----------------------------------------------------------
\section{Objetivos}
% ----------------------------------------------------------

Nas seções abaixo estão descritos o objetivo geral e os objetivos específicos deste TCC.

% ----------------------------------------------------------
\subsection{Objetivo Geral}
% ----------------------------------------------------------

O objetivo geral é desenvolver uma rede colaborativa de medidores de baixo custo para monitoramento da qualidade do ar de forma remota e em tempo real.

% ----------------------------------------------------------
\subsection{Objetivos Específicos}
% ----------------------------------------------------------

\begin{enumerate}
    \item Desenvolver os dispositivos de medição que serão adicionados a rede de monitoramento
    \item Disponibilizar uma \acrshort{api} para registro dos dados de monitoramento e acesso a eles em tempo real
    \item Desenvolver e disponibilizar bibliotecas de \textit{firmware} para o desenvolvimento dos dispositivos e sua comunicação com a \acrshort{api}
    \item Aplicar modelos de calibração a um equipamento e comparar seus resultados utilizando os dados de uma estação de monitoramento de referência
\end{enumerate}