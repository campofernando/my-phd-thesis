% ----------------------------------------------------------
\chapter{Introdução}
% ----------------------------------------------------------

A poluição do ar é um grande risco ambiental para a saúde, matando cerca de sete milhões de pessoas em todo o mundo todos os anos \cite{who2021}. Foram evidenciados pela Organização Mundial da Saúde \cite{who2021} os múltiplos danos da poluição do ar à saúde humana. O risco de sofrer doenças, ou mesmo de morte, é maior para as populações mais vulneráveis e marginalizadas, como as minorias raciais/étnicas e as pessoas de baixo nível socioeconômico \cite{Jbaily2022AirGroups}. Embora quase 99\% da população mundial respire ar que excede os limites das diretrizes da OMS \cite{who2021}, a cobertura da monitorização da qualidade do ar ainda é baixa e insuficiente \cite{Munir2019AnalysingSheffield}, especialmente em países de baixo e médio rendimento. Uma cobertura representativa da monitorização da poluição atmosférica é essencial para a gestão eficaz da qualidade do ar, dada a elevada variabilidade espaço-temporal da concentração de poluentes, especialmente em ambientes urbanos \cite{Kumar2015,Mead2013TheNetworks}.

Os poluentes atmosféricos têm sido monitorados usando equipamentos complexos e caros em locais fixos \cite{Kang2022PerformanceReview}. Os altos custos desses instrumentos limitam sua implantação a apenas algumas estações por cidade, deixando grandes áreas geográficas descobertas \cite{Munir2019AnalysingSheffield}, restringindo a resolução espacial e a distribuição das redes de monitoramento \cite{Jiao2016CommunityStates,Kumar2015}. No Brasil, por exemplo, as redes de monitoramento instaladas cobrem algumas cidades brasileiras, e seu desempenho a longo prazo é muitas vezes comprometido pela falta de manutenção e de pessoal qualificado \cite{Vormittag2021AnaliseBrasil}.

Buscando uma melhor compreensão do processo de poluição do ar e seus impactos, a ideia de sensoriamento onipresente atraiu a atenção da comunidade de qualidade do ar e o uso de monitores de baixo custo ganhou popularidade \cite{Motlagh2020TowardMonitoring,Kumar2015}. Devido à sua versatilidade e baixo custo (tanto de aquisição quanto de operação), esses dispositivos podem complementar a escassez espacial e temporal de redes certificadas de qualidade do ar e possibilitar novas aplicações de monitoramento \cite*{Lewis2016EvaluatingResearch}.

Diversas iniciativas têm sido realizadas como recursos para disponibilização de dados coletados em redes de monitoramento da qualidade do ar espacialmente mais densas, compostas por sensores de gases de baixo custo. Alguns fabricantes desses dispositivos de baixo custo, como AQMesh, i-Blades e Libelium fornecem, junto com suas plataformas de sensores, serviços de visualização e análise de dados, que são oferecidos como Software como Serviço (SaaS) ou Plataforma como Serviço ( PaaS). Outras iniciativas, lideradas por comunidades e instituições de investigação, forneceram recursos semelhantes às ferramentas de acesso aberto. Alguns deles são o mapa de Habitat \cite{HabitatMap2023AirCasting}, o Portal de Ovos de Qualidade do Ar \cite{AirQualityEgg2023AirPortal}, o mapa do projeto Luft Daten \cite{Sensor.Community2023LuftMap}, o Mapa PurpleAir \cite{PurpleAir2023PurpleAirMonitoring}, o Mapa do Cidadão Inteligente \cite{SmartCitizen2023SmartCitizen}, e o mapa da rede uRADMonitor \cite{uRADMonitor2023PM2.5URADMonitor}. O custo de aquisição das plataformas de sensores destas iniciativas varia entre 60,00 – 3.500,00 USD. Alguns deles (Luft Daten Project e Smart Citizen Map) fornecem guias para replicação dos dispositivos de monitoramento, de acordo com um conjunto de instruções de hardware e software, conforme estabelecido por seus criadores. Estes esforços estão a expandir as redes de monitorização da poluição atmosférica e a incorporar as comunidades e os cidadãos no debate global sobre temas de qualidade do ar, o que já é um passo essencial na democratização das aplicações de monitorização do ar. Contudo, a acessibilidade aos dispositivos ainda é limitada, especialmente nos países em desenvolvimento. Como a maioria destes instrumentos são comerciais, enquadram-se na categoria de “caixa preta” e podem ter custos elevados. Além disso, as atuais iniciativas de código aberto para a monitorização da qualidade do ar não conseguem dar resposta à grande variedade de aplicações potenciais de monitorização da qualidade do ar. Isto representa um problema para investigadores e promotores dispostos a implementar projetos de monitorização da qualidade do ar de baixo custo para registar e visualizar dados de poluição atmosférica.

Aqui, apresentamos a Rede Colaborativa de Qualidade do Ar Ambiental de Baixo Custo (CLEAN). A iniciativa inclui soluções de hardware (com documentação e tutoriais), bibliotecas de firmware bem documentadas e um aplicativo Web para visualização e análise de dados em tempo real. CLEAN facilita a incorporação de novos periféricos e sensores de hardware, independentemente dos requisitos da API, garantindo a interação com a aplicação Web independente do hardware desenvolvido para aplicações específicas de monitoramento.

CLEAN permite a colaboração de outros grupos e indivíduos interessados no desenvolvimento de dispositivos de monitoramento de baixo custo e dados abertos para análises ambientais. A plataforma web Renovar fornece uma API que permite que diversos dispositivos de monitoramento de baixo custo enviem seus dados para um servidor remoto para visualização e armazenamento em tempo real e geo-localizados. Além disso, a API possibilita a integração com outras aplicações Web para visualização e análise de dados. Esses dados permanecem abertamente disponíveis para posterior processamento e análise. Dada a grande versatilidade dos sensores de baixo custo, muitas aplicações poderiam ser monitoradas a partir de diversos cenários contribuindo para uma maior disponibilidade de volumes de dados.

% ----------------------------------------------------------
\section{Recomendações de uso}
% ----------------------------------------------------------

Este \emph{template} foi elaborado para se compilado em \LaTeX utilizando \abnTeX.  Todas as configurações de diferenciação gráfica nas divisões de seção e subseção seguem a  norma NBR 6027/2012 automaticamente. 

Uma nota de rodapé, já tem seu estilo automático com o comando \texttt{$\backslash$footnote}\footnote{As notas de rodapé possuem fonte tamanho 10. O alinhamento das linhas da nota de rodapé deve ser abaixo da primeira letra da primeira palavra da nota de modo dar destaque ao expoente.}.


% ----------------------------------------------------------
\section{Objetivos}
% ----------------------------------------------------------

Nas seções abaixo estão descritos o objetivo geral e os objetivos específicos deste TCC.

% ----------------------------------------------------------
\subsection{Objetivo Geral}
% ----------------------------------------------------------

Descrição...

% ----------------------------------------------------------
\subsection{Objetivos Específicos}
% ----------------------------------------------------------

Descrição...