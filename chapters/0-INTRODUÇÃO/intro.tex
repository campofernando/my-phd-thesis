
% \phantomsection
% ----------------------------------------------------------
\chapter*{Introdução}
% ----------------------------------------------------------
\phantomsection
\addcontentsline{toc}{chapter}{INTRODUÇÃO}%
\markboth{Introdução}{Introdução}%

A poluição do ar é de grande risco para a saúde, sendo responsável pela morte de cerca de sete milhões de pessoas em todo o mundo anualmente e de variadas doenças cardiorrespiratórias \cite{who2021}. Inclusive tem sido evidenciado que o risco de sofrer doenças, ou mesmo de morte, é maior nos grupos de população mais vulneráveis e marginalizadas, como as minorias raciais/étnicas e as pessoas de baixo nível socioeconômico \cite{Jbaily2022AirGroups}. Embora quase 99\% da população mundial respire ar que excede os limites das diretrizes da OMS \cite{who2021}, a cobertura de monitoramento da qualidade do ar ainda é baixa e insuficiente \cite{Munir2019AnalysingSheffield}, especialmente em países de baixo e médio rendimento \cite{Vormittag2021AnaliseBrasil,Ferreira2022}. Um monitoramento representativo da poluição atmosférica é essencial para a gestão eficaz da qualidade do ar, já que a variabilidade espaço-temporal da concentração de poluentes é elevada, especialmente em ambientes urbanos \cite{Kumar2015,Mead2013TheNetworks}.

Os poluentes atmosféricos têm sido monitorados usando equipamentos complexos e caros em locais fixos \cite{Kang2022PerformanceReview}. Os altos custos desses instrumentos limitam sua implantação a apenas algumas estações por cidade, deixando grandes áreas geográficas descobertas \cite{Munir2019AnalysingSheffield}, restringindo a resolução espacial e a distribuição das redes de monitoramento \cite{Jiao2016CommunityStates,Kumar2015}. No Brasil, por exemplo, as redes de monitoramento instaladas cobrem apenas aproximadamente 2\% do total de municípios brasileiras, e seu desempenho a longo prazo é muitas vezes comprometido pela falta de manutenção e de pessoal qualificado \cite{Vormittag2021AnaliseBrasil}.

Buscando uma melhor compreensão do processo de poluição do ar e seus impactos, a ideia de sensoriamento em alta resolução espacial tem atraído a atenção da comunidade de qualidade do ar, e o uso de monitores de baixo custo tem ganhado popularidade \cite{Motlagh2020TowardMonitoring,Kumar2015}. Devido à sua versatilidade - dada pelas suas pequenas dimensões e seu consumo energético reduzido \cite{Lewis2018Low-costApplications} - e seus custos baixos (tanto de aquisição quanto de operação), esses dispositivos podem complementar a escassez espacial e temporal de redes certificadas de qualidade do ar e expandir o horizonte de novas aplicações de monitoramento \cite{Lewis2016EvaluatingResearch}. 

Alguns fabricantes desses dispositivos de baixo custo, como AQMesh, Vaisala, i-Blades, Libelium e Clarity fornecem, junto com suas plataformas de sensores, serviços de visualização e análise de dados, que são oferecidos como Software como Serviço (SaaS), Plataforma como Serviço (PaaS) ou Sensoriamento como Serviço. Outras iniciativas de dados abertos, lideradas por comunidades e instituições de investigação, fornecem recursos semelhantes. Alguns deles são o \textit{Habitat Map} \cite{HabitatMap2023AirCasting}, o \textit{Air Quality Egg Portal} \cite{AirQualityEgg2023AirPortal}, \textit{Sensor.Community} \cite{Sensor.Community2023LuftMap}, o \textit{PurpleAir} \cite{PurpleAir2023PurpleAirMonitoring}, \textit{Smart Citizen} \cite{SmartCitizen2023SmartCitizen}, \textit{uRADMonitor} \cite{uRADMonitor2023PM2.5URADMonitor} e \textit{IQAir}. O custo de aquisição das plataformas de sensores destas iniciativas varia entre 60,00 – 3.800,00 USD. Alguns deles disponibilizam \acrshort{api}s para registro e acesso aos dados dos monitores mas na maioria dos casos o acesso é pago e condicionado à aquisição de monitores. Em geral, essas plataformas disponibilizam os dados para visualização de forma gratuita mas o restante dos serviços tem custos adicionais. Dentre elas, apenas \textit{Sensor.Community} e \textit{Smart Citizen} fornecem guias para replicação dos dispositivos de monitoramento. 

Da revisão realizada observou-se que as iniciativas de código aberto ainda são escassas. As que existem são direcionadas a aplicações específicas com arquiteturas que não buscam facilitar o seu reaproveitamento em outras aplicações de monitoramento. Especificamente no contexto brasileiro não foram identificados empreendimentos que fabricassem monitores de baixo custo próprios. Percebeu-se assim a necessidade de uma iniciativa que não apenas provesse \textit{software} e \textit{hardware} para determinada aplicação, mas que, através da modularização e o reaproveitamento de código e \textit{hardware}, facilitasse e acelerasse o desenvolvimento de novas aplicações.

Os monitores de baixo custo da qualidade do ar ainda devem alcançar níveis de confiabilidade apropriados para serem utilizados de forma estendida \cite{Penza2020Low-costMonitoring}. A literatura reporta diversos trabalhos que buscam reduzir o erro e a incerteza das medições dos sensores de baixo custo mediante a aplicação de modelos de calibração e de compensação \cite{Maag2018ADeployments, Concas2021LOW-COSTPREPRINT}. Dentre os modelos de calibração aplicados, as regressões multivariadas paramétricas e não paramétricas têm produzido os melhores resultados \cite{Feng2019ReviewTechnology,Concas2021LOW-COSTPREPRINT}. O motivo para o sucesso das regressões multivariadas está relacionado com a sua capacidade para considerar múltiplas variáveis. Isso têm sido vantajoso para os sensores de gases, cujas respostas sofrem pelas sensibilidades cruzadas com outros compostos \cite{Lewis2018Low-costApplications}, assim como pela influência das condições ambientais, como a temperatura \cite{Popoola2016DevelopmentStability} e a umidade relativa \cite{Pang2018TheMonitoring}. Particularmente, os modelos paramétricos que utilizam regressões lineares multivariadas (\gls{mlr}, siglas em inglês) \cite{Spinelle2015FieldDioxide} costumam produzir os melhores resultados quando os sensores são expostos a concentrações elevadas de poluentes \cite{Karagulian2019ReviewMonitoring}. Nesses cenários, a dinâmica linear entre a concentração e a saída dos sensores prevalece sobre as interferências de outras variáveis \cite{Hagan2018CalibrationInstruments}. Já nos ambientes com baixos níveis de concentração, as repostas dos sensores são mais sensíveis às não-linearidades introduzidas pelas variações da temperatura e da umidade \cite{Hagan2018CalibrationInstruments}. Nessas condições, os modelos não paramétricos superam o desempenho da \gls{mlr} \cite{Karagulian2019ReviewMonitoring}. Embora as regressões multivariadas, especialmente as não paramétricas, tenham alavancado o uso de monitores de baixo custo, ainda é necessário buscar modelos e metodologias de calibração e correção dos dados que se adaptem às condições e características de operação de cada aplicação de sensoriamento \cite{Concas2021LOW-COSTPREPRINT}.

Diante do exposto e pelas lacunas observadas, o presente trabalho descreve o desenvolvimento de uma iniciativa para facilitar a colaboração no desenvolvimento de redes de monitoramento de baixo custo e a sua aplicação num cenário real de monitoramento. Sendo assim, define-se como \textbf{objetivo geral} da tese desenvolver uma rede colaborativa de medidores de baixo custo para monitoramento da qualidade do ar. Para isso foram definidos como objetivos específicos:

\begin{enumerate}
    \item Desenvolver dispositivos de medição para serem adicionados à rede de monitoramento
    \item Disponibilizar uma \acrshort{api} para registro dos dados de monitoramento e acesso a eles em tempo real, assim como bibliotecas de \textit{firmware} para o desenvolvimento dos dispositivos e sua comunicação com a \acrshort{api}
    \item Identificar erros e interferências nas leituras de um equipamento de baixo custo e comparar as leituras com os dados de uma estação de monitoramento de referência
    \item Desenvolver uma metodologia de correção de leituras ruidosas ou em situação de falha de sensores
\end{enumerate}

A iniciativa desenvolvida, sob o nome de CLEAN (Collaborative Low-cost Environmental Air-quality Network), inclui uma \textit{api} aberta para registro e acesso de dados de monitores de baixo custo, bibliotecas de \textit{software} bem documentadas para o desenvolvimento do \textit{firmware} dos monitores e soluções de \textit{hardware} (com documentação e tutoriais) para determinadas aplicações. A \acrshort{api} permite que diversos dispositivos de monitoramento de baixo custo enviem seus dados geo-localizados para um servidor remoto para visualização e armazenamento em tempo real. Além disso, a API possibilita a integração com outras aplicações Web para visualização e análise de dados. Esses dados permanecem abertamente disponíveis para posterior processamento e análise. CLEAN também facilita a incorporação de novos periféricos e sensores de hardware, reutilizando o código das bibliotecas e garantindo a interação com a \acrshort{api} independentemente do \textit{hardware} desenvolvido para cobrir determinada aplicação de monitoramento. Dada a grande versatilidade dos sensores de baixo custo, muitas aplicações poderiam ser monitoradas a partir de diversos cenários contribuindo para uma maior disponibilidade de volumes de dados.

Os dispositivos de \textit{hardware} desenvolvidos para a rede foram dois protótipos de monitores de baixo custo e três equipamentos com placas de circuito impresso. Estes últimos são denominados de monitores CLEAN Arduino Mega para fins de versionamento e identificação. Os cinco dispositivos utilizam sensores de gases eletroquímicos (\gls{ec}) sensíveis aos poluentes \acrshort{co}, \acrshort{no2}, \acrshort{so2}, \acrshort{o3} e \acrshort{mp}. Foram escolhidos esses gases por serem gases de referência para indicar a qualidade do ar segundo definido na Resolução no 491/2018 do \gls{conama}. Os resultados obtidos com os dois protótipos foram apresentados no congresso internacional CMAS 2020 \cite{Campo2020DEPLOYMENTRESULTS} e encontram-se fora do escopo da presente tese. Um dos monitores CLEAN Arduino Mega foi instalado no município de Tubarão, no estado de Santa Catarina, junto a uma estação de referência, para análise e correção dos seus dados. São aplicadas metodologias de pré-processamento e correção dos dados utilizando modelos multivariados baseados em \gls{mlr}, \gls{knn}, \gls{rf}, \gls{ann}. Os resultados são comparados em termos de R2, erro, e correlação com as medições de referência. 

No Capítulo \ref{cap:air-quality-monitoring} apresenta-se uma revisão bibliográfica sobre o estado da arte no monitoramento da qualidade do ar, tanto o realizado para fins regulatórios como o de baixo custo, com especial foco no estado das redes de monitoramento no Brasil. No Capítulo \ref{cap:clean-initiative} é apresentada a iniciativa CLEAN e seus componentes, i.e.: a \acrshort{api}; as bibliotecas de \textit{firmware} e os dispositivos de \textit{hardware} desenvolvidos; e a aplicação \textit{Web} \textit{Renovar}. A iniciativa CLEAN foi também apresentada em artigo científico na revista Environmental Modelling \& Software \cite{Campo2021}. O código principal dos dispositivos de monitoramento e as bibliotecas de \textit{firmware} possuem registro de programa de computador no Instituto Nacional de Propriedade Industrial com número de registro BR512022001116-6. No Capítulo \ref{cap:field-monit-results} são apresentadas as medições realizadas pelo equipamento instalado em campo, as metodologia aplicadas para pré-processar os dados e corrigi-los utilizando modelos de regressão multivariados. Por último no Capítulo \ref{cap:calib-methodology} apresenta-se a metodologia aplicada para aprimorar os resultados obtidos no Capítulo \ref{cap:field-monit-results}.