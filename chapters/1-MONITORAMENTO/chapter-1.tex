% ----------------------------------------------------------
\chapter{Monitoramento da qualidade do ar}\label{cap:air-quality-monitoring}
% ----------------------------------------------------------

O monitoramento da poluição atmosférica se faz indispensável para o estudo e a gestão efetiva da qualidade do ar. O monitoramento considerado de referência é realizado por instituições governamentais através de redes de monitoramento automáticas \cite{Franca2019GUIAAR}. Essas redes são compostas por estações de monitoramento, certificadas por órgãos competentes, as quais registram ininterruptamente, e em tempo real, as concentrações dos poluentes na atmosfera \cite*{CETESB2020RedesAr}. As estações podem ser fixas ou móveis e a qualidade das suas leituras é garantida mediante procedimentos padrões de calibração dos instrumentos, de coleta de dados e de pós-processamento. As informações coletadas são processadas com base nos padrões legais estabelecidos e podem ser disponibilizados na forma de boletins diários ou relatórios anuais, como um resumo das condições da poluição atmosférica dentro de determinada área \cite{CETESB2020RedesAr}. No Brasil, para facilitar o entendimento e a divulgação da informação da qualidade do ar de curto prazo, é utilizado o Índice da Qualidade do Ar \gls{iqar}. Este índice é um valor adimensional que outorga uma nota de Boa até Péssima à qualidade do ar, relacionando a informação quantitativa de concentração de determinados poluentes com seus efeitos na saúde humana \cite{Franca2019GUIAAR}. Os poluentes que são regulados pela legislação brasileira, na Resolução no. 491/2018, e que portanto são medidos pelas redes de monitoramento de referência são: o material particulado (\acrshort{mp}), as partículas totais em suspensão, o dióxido de enxofre (\acrshort{so2}), o dióxido de nitrogênio (\acrshort{no2}), o ozônio ao nível de solo (\acrshort{o3}), o monóxido de carbono (\acrshort{co}) e a fumaça.

As redes de monitoramento de referência caracterizam-se por uma elevada precisão e confiabilidade. Contudo, fatores como preço, consumo energético, complexidade de operação e dimensões acabam dificultando seu uso de forma mais extensa. A literatura reporta preços dessas tecnologias em torno de \$ 15.000 – \$ 100.000 \cite{Concas2021LOW-COSTPREPRINT} e consumo de potência aproximadamente entre 0.2 e 1 kW \cite{Piedrahita2014TheMonitoring}, sem contar os custos advindos dos procedimentos de manutenção \cite{Kumar2015}. Sendo assim, a onerosidade da solução como um todo limita o número de estações financeiramente viáveis e reduz a resolução e distribuição espacial das medições, inclusive em países e regiões com elevados índices de desenvolvimento econômico como os Estados Unidos e a Europa \cite*{Kumar2015,Jiao2016CommunityStates}.

No Brasil o monitoramento de referência é ainda bastante limitado. Algumas iniciativas governamentais e de órgãos de pesquisa, a nível nacional e estadual, têm sido implementadas para facilitar o acesso a dados sobre a qualidade do ar como, por exemplo: \cite{CETESB2020RedesAr}, \cite{IEMA2020QualidadeAr} e \cite{IEMA/ES2020IEMAAr}. No entanto, ainda assim, as redes de monitoramento que têm sido instaladas cobrem escassos pontos das cidades brasileiras e seu desempenho a longo prazo muitas vezes se vê comprometido por falta de manutenção e de pessoal qualificado \cite{Oyama2017AIRBRAZIL}. Segundo um estudo de 2019 realizado pelo Instituto Saúde e Sustentabilidade, das 27 unidades federativas apenas 7 monitoram a qualidade do ar e atendem o regulamento vigente. O restante das unidades não realizam o monitoramento ou o realizam de forma ineficiente \cite{InstitutoSaudeeSustentabilidade2019AnaliseBrasil}. O estudo também aponta que das 319 estações ativas no país, a região sudeste concentra o 93\% delas. O estado de Santa Catarina, em particular, não realiza monitoramento da qualidade do ar.

A pesar da sua elevada confiabilidade, devido à sua limitada distribuição espacial, as redes de monitoramento de referência da qualidade do ar proveem apenas informações globais relativas a níveis de concentrações de fundo, o que dificulta a compilação de informação confiável e representativa de áreas específicas \cite{Kumar2015}. Esse fato constitui uma limitante para o estudo dos processos associados à poluição atmosférica. Dada a alta variabilidade espaço-temporal da concentração dos poluentes, especialmente em ambientes urbanos \cite{Mead2013TheNetworks}, a resolução espacial das redes de monitoramento é tão relevante quanto a confiabilidade das suas estações \cite{Jiao2016CommunityStates}.

Diante disso, se faz necessário a busca de novas soluções que possibilitem incrementar o número de monitores viáveis nas redes de monitoramento sem afetar a qualidade das medições. Todavia, especial cuidado deve ser tomado para não comprometer a qualidade dos dados obtidos, já que, como apontado por Emily Snyder e colaboradores, ter dados pouco confiáveis é mais prejudicial do que não ter dados, pois podem conduzir a decisões desacertadas \cite{Snyder2013}.

Com o intuito de regulamentar o uso de monitores de baixo custo segundo a confiabilidade das suas leituras, a Diretiva Europeia para a Qualidade do Ar definiu o Objetivo da Qualidade dos Dados (\gls{dqo}) como requisito necessário para o uso destes instrumentos para fins de indicação da concentração de poluentes atmosféricos \cite{EU2008DirectiveEurope}. O \gls{dqo} representa o nível de incerteza aceitável nas medições destes instrumentos sendo de 50\% para \acrshort{mp}, 30\% para \acrshort{o3} e 25\% para \acrshort{co}, \acrshort{nox} e \acrshort{so2}. A Agência de Proteção Ambiental Norte-americana (\gls{epa}), por outro lado, sugere a avaliação dos instrumentos de monitoramento considerando cinco áreas de aplicação (Tabela \ref{tab:monitoring-areas-EPA}). Cada área tem valores de precisão, viés e completude dos dados que devem ser cumpridos para que um monitor de baixo custo possa ser utilizado em aplicações dentro dessa área \cite{Williams2014AirGuidebook}. Numa direção um pouco diferente dos dois órgãos anteriores, em um estudo mais recente conduzido por Lidia Morawska, foi proposto que, dado o amplo leque de aplicações de monitoramento, os requisitos de desempenho dos monitores de baixo custo fossem definidos de acordo com cada aplicação prescindindo assim de métricas padrões \cite{Morawska2018ApplicationsGone}. Esta última abordagem, contudo, exige um conhecimento profundo da aplicação e dos requerimentos de desempenho associados \cite{Morawska2018ApplicationsGone}.

\begin{table}[t]
    \caption{Requerimentos de desempenho dos instrumentos de monitoramento da qualidade do ar segundo área de aplicação}
    \begin{tabular}{ 
    | m{0.35\textwidth} m{0.3\textwidth} m{0.10\textwidth} m{0.15\textwidth} | }
        \hline
        Área de aplicação & Poluentes & Erro & Completude dos dados \\ [0.5ex] 
        \hline
        Educação e Informação & Todos & < 50\% & $\geq$ 50\% \\ [0.5ex]
        \hline
        Identificação e Caracterização de \textit{Hotspots}\footnotemark & Todos & < 30\% & $\geq$ 75\% \\ [0.5ex]
        \hline
        Monitoramento Complementar & Todos os regulados pelo \gls{conama} e os \acrshort{voc} & < 20\% & $\geq$ 80\% \\ [0.5ex]
        \hline
        Exposição pessoal & Todos & < 30\% & $\geq$ 80\% \\ [0.5ex]
        \hline
        Monitoramento de referência & \acrshort{o3} \newline \acrshort{co}, \acrshort{so2} \newline \acrshort{no2} \newline \acrshort{mp} & < 7\% \newline < 10\% \newline < 15\% \newline < 10\% & $\geq$ 75\% \\ [0.5ex]
        \hline
    \end{tabular}
    \label{tab:monitoring-areas-EPA}
    \fonte{\cite{Williams2014AirGuidebook}}
\end{table}
\footnotetext{Pontos de elevada concentração de determinado poluente}

% ----------------------------------------------------------
\section{Importância}
% ----------------------------------------------------------

Importância (Efeitos da poluição atmosférica e necessidade de monitoramento para controle)

% ----------------------------------------------------------
\section{Monitoramento de referência}
% ----------------------------------------------------------

(custos, equipamentos, técnicas)

% ----------------------------------------------------------
\section{Monitoramento de baixo custo da qualidade do ar}\label{section:low-cost-air-quality-monit}
% ----------------------------------------------------------

% (estado atual, vantagens, desvantagens e desafios, Brasil)

Segundo Emily Snyder e colaboradores  o monitoramento da qualidade do ar tem experimentado uma mudança de paradigma na forma como os dados são coletados \cite{Snyder2013}. Os avanços recentes em instrumentação eletrônica unido a necessidade de soluções alternativas que complementem as técnicas de monitoramento tradicionais, têm contribuído para um crescente interesse no desenvolvimento de sensores de qualidade do ar de baixo custo \cite{Kumar2015,Lewis2018Low-costApplications}. Esses novos sensores possuem características essenciais como tamanho e peso reduzidos, baixo consumo de potência, baixo custo e facilidade de uso, que os colocam em vantagem com relação aos instrumentos de referência, e que têm criado as condições para aprimorar uma série de aplicações de monitoramento e gerar novas \cite{Snyder2013,Lewis2018Low-costApplications}.

Este tipo de tecnologia possibilitaria a agências públicas, entidades reguladoras e de pesquisa utilizar um maior volume de sistemas de monitoramento, e assim diversificar e complementar as aplicações de monitoramento para fins de pesquisa e regulamentação e validar modelos atmosféricos \cite{Lewis2018Low-costApplications}. Estes sistemas podem prover informação temporal qualitativa relevante sobre o nível de poluição atmosférica em uma determinada localidade por períodos de dias a meses \cite{Castell2018LocalizedNodes}, como por exemplo, os momentos do dia em que a poluição é maior ou menor \cite{Zimmerman2018AMonitoring}, ou observar a sua variação ao longo do tempo \cite{Castell2017CanEstimates}. Igualmente, os sistemas de baixo custo têm sido utilizados para detectar de áreas com níveis de concentração elevados nas cidades \cite{Mead2013TheNetworks} e elaborar mapas de poluição \cite{Huang2019EstimatingMeasurements}.

Para Emily Snyder e colaboradores o uso deste tipo de sensores pode levar a uma melhor proteção da saúde pública e do meio ambiente \cite{Snyder2013}. Sensores de baixo custo podem ser usados para prover informação representativa de exposição pessoal a poluentes, com elevada resolução temporal \cite{Mead2013TheNetworks,Jerrett2017ValidatingScience}. Igualmente, esta nova forma de medição, por ser de baixo custo e de fácil operação, pode ser instalada em comunidades \cite{Mahajan2020APollution}, escolas e áreas residenciais \cite{Castell2018LocalizedNodes}, provendo informação à população sobre a qualidade do ar que respiram, e colocando os processos de monitoramento e medição nas mãos de comunidades e indivíduos \cite{Lewis2018Low-costApplications}.

Segundo seu princípio de funcionamento, os sensores de baixo custo utilizados para a medição de poluentes na atmosfera podem ser classificados em: sensores de material particulado e sensores de gases \cite{Maag2018ADeployments}. Os sensores de material particulado funcionam baseados em princípios  de detecção óticos que medem o espalhamento ou a adsorção da luz pelas partículas \cite{Rai2017End-userMonitoring}. Já os sensores de gases podem ser subdivididos em duas classes: os óticos e os que dependem da interação entre o material transdutor e o composto gasoso \cite{Snyder2013}.
	
O princípio ótico de medição consiste em expor o composto gasoso a um feixe de luz, com determinado comprimento de onda, e medir o efeito dessa interação com um detector fotossensível \cite{Rai2017End-userMonitoring}. Dentro desse grupo, os sensores mais comuns são os infravermelhos não-dispersivos (\gls{ndir}, por suas siglas em inglês), usados para medir \acrshort{co2} e \acrshort{ch4}, e os detectores de foto-ionização (\gls{pid}, por suas siglas em inglês) que utilizam luz ultravioleta e são usados para medir compostos orgânicos voláteis \cite{Snyder2013}.

Dentre os sensores de gases que operam a partir da interação entre o material transdutor e o composto gasoso, os mais populares são: os semicondutores de óxido metálico (\gls{mos}, por suas siglas em inglês) e os de princípio eletroquímico (\gls{ec}, por suas siglas em inglês). Esses são os sensores mais comumente utilizados para medir gases tóxicos como \acrshort{co}, \acrshort{nox}, \acrshort{o3} e \acrshort{so2} \cite{Lewis2018Low-costApplications}.

Os sensores \gls{ec}, em comparação com os \gls{mos}, costumam ter um menor consumo de potência, maior seletividade, menores limites de detecção e uma relação linear com a concentração. Os sensores \gls{mos}, por sua parte, têm custos menores e seu condicionamento eletrônico costuma ser mais simples \cite{Rai2017End-userMonitoring}. Neste trabalho serão abordados apenas os sensores eletroquímicos por serem os mais utilizados para o monitoramento de baixo custo da qualidade do ar e pela relação linear entre suas respostas e a concentração de gás.

É importante ressaltar que o atual estado da arte dos sensores de baixo custo impossibilita que eles substituam os métodos de medição de referência. Contudo, resultados promissores têm sido encontrados em várias áreas, principalmente naquelas em que as técnicas convencionais não poderiam ser implementadas devido a limitações como a pouca portabilidade, o elevado consumo energético e o custo. 

% ----------------------------------------------------------
\subsection{Limitações do monitoramento de baixo custo da qualidade do ar}\label{subsection:low-cost-monit-limits}
% ----------------------------------------------------------

Os sensores de gases de baixo custo, como qualquer sistema de medição, possuem fontes de erro internas que são inerentes ao seu próprio funcionamento \cite{Maag2018ADeployments}. Estes erros são geralmente conhecidos e fáceis de determinar. Além desses, existem fontes de interferência externas, relacionadas às suas condições de operação, que são mais difíceis de detectar e controlar. Inclusive, sensores de um mesmo fabricante e de um mesmo lote de fabricação, podem apresentar comportamentos diferentes perante a influência de fontes externas \cite{Alphasense2013AlphasenseHUMIDITY,Castell2017CanEstimates}.

Os fabricantes normalmente definem um intervalo de medição onde os sensores apresentam melhor desempenho. O limite inferior desse intervalo é conhecido como limite de detecção, e todos os valores inferiores a ele são considerados como ruído \cite{Maag2018ADeployments}. Esse parâmetro é determinado em condições de laboratório, por isso em condições de operação não controladas, o seu valor pode sofrer alterações levando a erros nas medições. Por exemplo, em ambientes com muitas fontes de ruído electromagnético, ou um circuito de alimentação de energia pouco robusto a flutuações na tensão elétrica, podem aumentar a amplitude do ruído elétrico e, com ele, o limite de detecção dos sensores, afetando a sua resolução.

Outro tipo de erro sistemático que se manifesta internamente são os erros de offset e de sensibilidade. Como estes erros são não aleatórios são relativamente fáceis de remover mediante calibrações de laboratório \cite{Spinelle2013ProtocolPollution}. As derivas são outra fonte interna de erros produto de alterações na sensibilidade dos sensores devido principalmente ao seu envelhecimento, que dificultam seu uso para monitoramento a longo prazo \cite{Feng2019ReviewTechnology}. As derivas podem ser eliminadas com re-calibrações periódicas. A frequência das re-calibrações depende do sensor e da concentração de poluentes a que é exposto, podendo chegar a ser quinzenal \cite{Concas2021LOW-COSTPREPRINT}.

Fatores externos ao funcionamento dos sensores também produzem interferências nas medições. A influência desses fatores, principalmente das condições ambientais, são identificados por grande parte dos autores como um dos principais desafios no tratamento das respostas dos sensores \cite*{Mead2013TheNetworks,Popoola2016DevelopmentStability,Rai2017End-userMonitoring,Baron2017AmperometricReview}. Esses problemas são característicos de medições feitas em ambientes externos, em condições e ambientes reais, não controladas, ao contrário das medições tomadas em condições de laboratório sob as quais o desempenho dos sensores costuma ser muito melhor \cite{Castell2017CanEstimates}.

Variações na temperatura e na umidade do ambiente afetam a sensibilidade e o valor de linha base dos sensores \cite{Popoola2016DevelopmentStability,Pang2018TheMonitoring}. Particularmente, tem sido observado que em ambientes externos, onde os níveis de concentração costumam encontrar-se na ordem dos \gls{ppb}, as variações na temperatura e na umidade relativa alteram o valor de linha base em maior medida que a sensibilidade \cite{Popoola2016DevelopmentStability}.

Os fabricantes de sensores muitas vezes disponibilizam informações sobre a relação entre as respostas dos sensores e as variáveis ambientais, junto a modelos lineares de compensação \cite{Alphasense2013AlphasenseHUMIDITY,SPECSensors2016ApplicationPerformance}. No entanto, essas informações são obtidas a partir de testes de laboratório que simulam condições reais \cite{Spinelle2013ProtocolPollution}, sendo válidas apenas em níveis de concentração na ordem dos \gls{ppm} e em condições similares às dos testes \cite{Lewis2018Low-costApplications}. Dessa forma, as soluções para compensar os efeitos da temperatura e a umidade relativa providas pelos fabricantes resultam insuficientes para aplicações em campo \cite{Pang2018TheMonitoring}. Essas informações são especialmente limitadas para aplicações de monitoramento móvel, onde os sensores são expostos a transientes de concentração bruscos e condições ambientais variadas \cite{Delaine2019InReview}.
	
Uma solução que o fabricante Alphasense tem aplicado nos seus sensores é a incorporação de um quarto eletrodo, chamado eletrodo auxiliar, cujo sinal de saída é utilizado para compensar os efeitos de variáveis ambientais no valor de linha base do sensor \cite{Alphasense2019AlphasenseSensors}. Este eletrodo tem uma composição similar ao eletrodo de trabalho e provê um sinal de corrente de linha base que acompanha as variações do eletrodo de trabalho decorrentes das mudanças na temperatura, a umidade relativa e a pressão, podendo ser subtraída para obter a resposta do sensor ao gás \cite{Baron2017AmperometricReview}. Idealmente deveria funcionar assim, contudo, na prática tem sido demonstrado que o eletrodo auxiliar não é capaz de acompanhar as variações do eletrodo de trabalho em todo o intervalo de temperaturas de operação, e que portanto, uma simples subtração é insuficiente para gerar uma resposta confiável \cite{Wei2018ImpactMonitoring}. Cross e colaboradores também comprovaram que as correções recomendadas pelo fabricante Alphasense utilizando o eletrodo auxiliar não produzem os níveis de acurácia requeridos nas medições em ambientes externos \cite{Cross2017UseMeasurements}.
	
Tem sido reportado que variações bruscas na umidade relativa e na pressão ambiente produzem picos nas respostas dos sensores que invalidam as leituras por intervalos de tempo de até 40 minutos \cite{Alphasense2013AlphasenseHUMIDITY,Lewis2018Low-costApplications}. Igualmente ambientes com valores de umidade muito extremos ou muito poluídos podem saturar os sensores ocasionando falhas e reduzindo sua sensibilidade \cite{Alphasense2013AlphasenseHUMIDITY}.

Outros fatores que influenciam no desempenho dos sensores são as variações nos níveis de concentração do local de instalação. Os sensores eletroquímicos, por exemplo costumam ter melhor desempenho em locais onde os níveis de poluição são elevados \cite{Castell2017CanEstimates}, já que nestas condições a dinâmica senso-gás predomina sobre o efeito das variáveis interferentes \cite{Hagan2018CalibrationInstruments}. Por exemplo, Hagan e colaboradores comprovaram que o efeito da umidade relativa poderia ser ignorado ao calibrar um sensor \gls{ec} sensível a \acrshort{so2} em todo o intervalo de medição do sensor, contudo, para medições abaixo de 25 ppb, o efeito da umidade relativa mostrou-se significativa \cite{Hagan2018CalibrationInstruments}. Nuria Castell e colaboradores obtiveram coeficientes de correlação maiores com sensores instalados em locais com trânsito intenso do que nos locais com trânsito leve \cite{Castell2017CanEstimates}. Eles também constataram que o coeficiente de correlação dos sensores testados em locais com trânsito intenso caiu de forma considerável durante o período de férias, quando o trânsito pelo local foi reduzido \cite{Castell2017CanEstimates}. Por esse motivo, outro dos desafios dos monitores de baixo custo é que os instrumentos mantenham um bom desempenho independentemente dos níves de concentração encontrados no local \cite{Concas2019ACalibration}.

Outro problema comum dos sensores de baixo custo é a sensibilidade cruzada, que é a sensibilidade que os sensores têm a outros gases além do gás de interesse \cite{Maag2018ADeployments}. Por exemplo, é bem conhecido que os sensores de \acrshort{o3} são também sensíveis ao \acrshort{no2} \cite{Pang2017ElectrochemicalMonitoring,Alphasense2019AlphasenseSensors}. Outros estudos têm encontrado também sensibilidade a \acrshort{no2} em sensores de \acrshort{no} e \acrshort{so2} \cite{Lewis2016EvaluatingResearch}, assim como a \acrshort{o3} e \acrshort{co2} em sensores de \acrshort{no2} \cite{Lewis2018Low-costApplications}. Este efeito é especialmente desvantajoso em ambientes externos onde o ar é formado por uma mistura complexa de compostos gasosos. Por isso, se faz necessário assegurar que a leitura de um sensor corresponda ao gás para o qual foi projetado sem a interferência de outros compostos.

Uma forma de abordar o problema da sensibilidade cruzada é otimizando o material do eletrodo de trabalho durante a fabricação do sensor, para facilitar ou catalisar apenas as reações do gás de interesse \cite{R.Stetter2008AmperometricReview}. Igualmente, a seletividade pode ser melhorada no circuito de condicionamento, fixando o potencial de trabalho em um valor que favoreça as reações para o gás objeto de estudo \cite{R.Stetter2008AmperometricReview,Alphasense2013AlphasenseWork}. Em ambientes externos essas abordagens costumam ser insuficientes, sendo necessário utilizar arranjos de sensores nas medições e aplicar técnicas de calibração multivariadas que considerem a resposta global do arranjo \cite{Maag2018ADeployments}.

Dada a multiplicidade e complexidade dos fatores que influenciam as medições dos sensores de baixo custo, se faz necessário o estudo cuidadoso e a aplicação de técnicas de calibração que garantam níveis de precisão e confiabilidade aceitáveis para cada aplicação de monitoramento. Isso é de importância especialmente nas aplicações de monitoramento móvel, em que os sensores são expostos a transientes bruscos nas condições de operação.