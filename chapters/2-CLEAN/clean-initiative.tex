% ----------------------------------------------------------
\chapter{CLEAN - Collaborative Low-cost Environmental and Air-quality Network}\label{cap:clean-initiative}
% ----------------------------------------------------------

A iniciativa CLEAN consiste numa plataforma colaborativa para promover e facilitar o desenvolvimento de monitores de qualidade do ar de baixo custo e o acesso remoto a informações sobre a qualidade do ar em tempo real. Possui quatro componentes principais: i) dispositivos de hardware ii) firmware reutilizável e bibliotecas de Programação Orientada a Objetos baseadas no framework Arduino para programação dos dispositivos de monitoramento, iii) guias e documentação para reprodução do hardware e adesão à rede, iv) e a plataforma Web e API Renovar para visualização, registro e acesso remoto de dados em tempo real. Todos os guias e documentação relativos ao desenvolvimento do hardware e firmware dos dispositivos até agora concebidos, as bibliotecas implementadas e as ferramentas de desenvolvimento estão abertas e disponíveis gratuitamente na página inicial de CLEAN \cite{Campo2021}.

CLEAN permite a colaboração de outros grupos e indivíduos interessados no desenvolvimento de dispositivos de monitoramento de baixo custo e dados abertos para análises ambientais. A plataforma web Renovar fornece uma API que permite que diversos dispositivos de monitoramento de baixo custo enviem seus dados para um servidor remoto para visualização e armazenamento em tempo real e geo-localizados. Além disso, a API possibilita a integração com outras aplicações Web para visualização e análise de dados. Esses dados permanecem abertamente disponíveis para posterior processamento e análise. Dada a grande versatilidade dos sensores de baixo custo, muitas aplicações poderiam ser monitoradas a partir de diversos cenários contribuindo para uma maior disponibilidade de volumes de dados. A Figura 1 ilustra os principais componentes da iniciativa CLEAN, que serão descritos a seguir.

Possui quatro componentes principais: 

\begin{enumerate}
    \item Firmware reutilizável e bibliotecas de Programação Orientada a Objetos baseadas no framework Arduino para programação dos dispositivos de monitoramento
    \item A aplicação web \href{http://renovar.lcqar.ufsc.br/}{Renovar} para visualização (e acesso) remoto dos dados em tempo real
    \item Dispositivos de hardware
    \item Guias e documentação para reprodução do hardware
\end{enumerate}

% ----------------------------------------------------------
\section{Bibliotecas de firmware}
% ----------------------------------------------------------

O \textit{firmware} dos dispositivos foi desenvolvido no \textit{Framework Arduino}, que é uma abstração de códigos-fonte e bibliotecas comuns a diversas plataformas de \textit{\textit{hardware}}. O \textit{framework} fornece bibliotecas de código escritas em C/C++ para programação de microcontroladores e interação com dispositivos periféricos. Esta estrutura torna possível escrever programas para controlar uma ampla gama de placas microcontroladoras de Arduino e de outros fabricantes.

Para programar todas as funcionalidades do \textit{firmware} CLEAN, o código foi estruturado em um conjunto de classes em C++ separados em módulos ou bibliotecas. Esta estrutura foi concebida visando sua reutilização em outras plataformas de microcontroladores e outros componentes de \textit{hardware} suportados no Framework Arduino (como ESP8266 da Espressif) e também para facilitar a revisão e manutenção do código. Para melhor compreender a utilidade desta estrutura comparem-se os diagramas da Figura \ref{fig:application-structure}.

\begin{figure}[t]
    \centering
    \caption{Estrutura de desenvolvimento de aplicação de monitoramento de baixo custo}
    \begin{subfigure}{0.40\textwidth}
        \includegraphics[width=\textwidth]{chapters/2-CLEAN/Figuras/Camadas aplicação genérica (PT).png}
        \caption{Aplicação genérica}
        \label{fig:application-structure-generic}
    \end{subfigure}
    \hfill
    \begin{subfigure}{0.40\textwidth}
        \includegraphics[width=\textwidth]{chapters/2-CLEAN/Figuras/Camadas aplicação CLEAN (PT).png}
        \caption{Aplicação CLEAN}
        \label{fig:application-structure-clean}
    \end{subfigure}
    \hfill
    \label{fig:application-structure}
    \fonte{Desenvolvido pelo autor (2023)}
\end{figure}

A Figura \ref{fig:application-structure-generic} representa a estrutura de desenvolvimento de uma aplicação genérica de monitoramento. Para a criação de um monitor de qualidade do ar de baixo custo que envie dados para a \acrshort{api} requer-se de um dispositivo físico composto por uma camada de \textit{hardware}, com sensores e periféricos, e uma camada de \textit{firmware} que realiza o interfaceamento do elementos de \textit{hardware}, a aquisição dos dados e sua transmissão para a \acrshort{api}. Seguindo essa topologia, cada nova aplicação que for abordada precisaria de novas iterações pelas mesmas etapas de criação e desenvolvimento com mínimo reaproveitamento de iterações anteriores.

A Figura \ref{fig:application-structure-clean} ilustra a solução proposta pela iniciativa CLEAN. O intuito é desacoplar o que é específico de cada aplicação e o que é comum para todas as aplicações que visem utilizar a \acrshort{api} Renovar para registro dos dados de monitoramento. Dessa forma, os componentes de \textit{hardware} e o \textit{firmware} para interfacear estes componentes ficariam sob responsabilidade dos desenvolvedores das aplicações, enquanto as funcionalidades relacionadas à aquisição de dados e comunicação com a \textit{api} estariam cobertas dentro do conjunto de classes e funcionalidades disponibilizadas pelas bibliotecas de CLEAN. Dessa forma maximiza-se o reaproveitamento de código minimizando o esforço para implantar novas soluções de monitoramento.

As classes desenvolvidas para o projeto estão distribuídas em quatro módulos principais, conforme mostrado na Figura \ref{fig:fw-libraries-structure}: o módulo \textit{Hardware Interfaces}, o módulo \textit{System Drivers}, o módulo \textit{}{Sensors} e o módulo \textit{Data}.

O módulo \textit{Hardware Interfaces} agrupa todas as classes e estruturas utilizadas para se comunicar com o \textit{hardware} periférico, como sensores, módulos de temporização, módulos de geolocalização e módulos de armazenamento. Os \textit{Drivers} implementam funcionalidades que podem ser utilizadas pelo programa principal independentemente do \textit{hardware} utilizado em cada dispositivo. O pacote \textit{Sensors} está no mesmo nível dos \textit{Drivers} e pode ser interpretado como um conjunto de \textit{drivers} especiais para os sensores, mas com a particularidade de ser específico para cada fabricante. Por fim, o pacote Dados engloba todas as funcionalidades relacionadas à preparação de dados de sensores para armazenamento e transmissão. Este pacote abstrai as informações de concentração adquiridas pelos sensores de gás a partir de detalhes específicos sobre o funcionamento e operação de seu \textit{hardware}.

\begin{figure}[h]
    \centering
    \caption{Conjunto de bibliotecas utilizadas para \textit{firmware} dos dispositivos CLEAN}
    \includegraphics[width=0.75\linewidth]{chapters//2-CLEAN/Figuras/Diagrama de bibliotecas.png}
    \label{fig:fw-libraries-structure}
\end{figure}

\subsection{O módulo de interfaces de \textit{hardware}}

As Interfaces de \textit{Hardware} abrangem as funcionalidades relacionadas à comunicação e interface de sensores de gás, módulos de geoposicionamento (\acrshort{gps}) e relógio de tempo real (\acrshort{rtc}) que foram utilizados nos equipamentos desenvolvidos. O modo de operação e a saída dos sensores e de cada dispositivos de \textit{hardware} determinarão o seu esquema de conexão ao microcontrolador e a forma como sua leitura é implementada no \textit{firmware}.

A Figura \ref{fig:fw-libraries-hw-interfaces} mostra um diagrama das classes que foram implementadas para a versão atual do \textit{firmware}. As classes \texttt{SerialSensorInterface} e \texttt{AnalogSensorInterface} implementam interfaces para sensores digitais e analógicos, respectivamente. \texttt{SerialSensorInterface}, em particular, implementa uma interface para um sensor digital conectado através de um barramento \acrshort{uart} ou RS-485, por meio das classes filhas \texttt{UARTSensorInterface} e \texttt{RS485SensorInterface}. 

Cada classe implementa seu próprio método \texttt{sense()} que recebe como parâmetros um ponteiro para um objeto \texttt{Stream} (geralmente uma porta serial do microcontrolador), e um ponteiro para um \texttt{SerialParser}, que analisa as cadeias de caracteres com comandos ou dados enviados pelo sensor digital. O \texttt{SerialParser} é implementado numa camada superior pelas classes do módulo \texttt{Sensor}.

\begin{figure}[h]
    \centering
    \caption{Diagramas de classes do pacote \textit{Hardware} Interfaces}
    \includegraphics[width=0.90\linewidth]{chapters//2-CLEAN/Figuras/Diagrama-de-classes-Hardware-Interfaces.png}
    \label{fig:fw-libraries-hw-interfaces}
\end{figure}

A interface com um dispositivo serial para conexão à internet foi implementada através das classes \texttt{SerialInternetInterface} e \texttt{ESPSerialInterface}, esta última representando a conexão com o microcontrolador ESP8266. Foram criadas mais duas interfaces para módulos \acrshort{gps} e \acrshort{rtc}. Na versão atual do \texttt{firmware}, foram utilizadas as bibliotecas \texttt{TinyGPSPlus} e \texttt{RtcDS3221} para cada módulo respectivamente, porém, qualquer outra biblioteca ou módulo também pode ser usado, desde que seja criado como uma classe filha de \texttt{GPSSerialInterface} e \texttt{RTCInterface}. Para isso, as classes filhas deverão implementar os métodos virtuais: \texttt{readGPS()}, \texttt{set\_time()} e \texttt{get\_time()} respectivamente.

\subsection{O módulo \textit{drivers}}

Os \textit{Drivers} atuam como uma camada intermediária entre as Interfaces de \textit{Hardware} e o programa principal. Eles abstraem o \textit{hardware} dos dispositivos do código principal, permitindo sua reutilização independentemente dos módulos e bibliotecas utilizadas em um nível inferior. Alguns \textit{drivers} implementados para o \textit{firmware} foram:

\begin{itemize}
    \item O driver \texttt{HardStorage}, para armazenamento de dados em cartão \textit{SD}; 
    \item O \texttt{RTCDriver} para a Interface \acrshort{rtc};
    \item O \texttt{GPSDriver} para a Interface \acrshort{gps}; 
    \item O \texttt{TimeDriver} para gerenciamento das fontes de tempo no dispositivo, as quais podem provir de um módulo \acrshort{rtc}, um módulo \acrshort{gps} ou de um servidor \acrshort{ntp}
\end{itemize}

Esses quatro \textit{drivers} usam métodos estáticos, o que significa que podem ser usados sem necessidade de ter um objeto implementado no código. 

Os outros dois \textit{drivers} que foram implementados estão relacionados ao tratamento dos dados. São eles o \texttt{DataContainer} e o \texttt{Smoother}. A Figura \ref{fig:fw-libraries-drivers} mostra o diagrama de classes deste módulo. A continuação são resumidos alguns dos principais métodos e atributos de cada uma das classes pertencentes ao módulo \texttt{Drivers}.

\begin{figure}[h]
    \centering
    \caption{Diagrama de classes do Módulo \textit{Drivers}}
    \includegraphics[width=0.90\linewidth]{chapters//2-CLEAN/Figuras/Diagrama-de-classes-System-Drivers-1.png}
    \label{fig:fw-libraries-drivers}
\end{figure}

\subsubsection{TimeDriver}
Esta classe registra a data e hora internas do sistema e fornece métodos para retornar informações de data e hora em diferentes formatos. O método \texttt{set\_time(time\_t)} define a data e hora do sistema. Internamente, ele invoca o método \texttt{setTime()} da biblioteca \texttt{Time.h} do framework Arduino. Recebe como parâmetro um número inteiro de 32 bits contendo a data e hora fornecidas por alguma fonte de relógio externa (um módulo \acrshort{gps}, um módulo \texttt{rtc} ou um servidor \texttt{ntp}).
\subsubsection{GPSDriver}
Esta classe controla a interface com um módulo \texttt{gps}, armazena as informações das coordenadas geográficas do sistema e fornece métodos para acessá-las, sendo eles:

\begin{itemize}
    \item \texttt{{static} get\_latitude(): double}
    \item \texttt{{static} get\_longitude(): double}
    \item \texttt{{static} get\_altitude(): double}
    \item \texttt{{static} get\_gps\_st(): GPSSt\_e}
\end{itemize}

Esses métodos fornecem as informações de geolocalização armazenadas no \texttt{GPSDriver}, bem como o estado dessas informações. As informações de geolocalização podem estar OK ou desatualizadas. Esses dois valores são retornados como uma enumeração do tipo \texttt{GPSSt\_e}.

Outros dois métodos definem as coordenadas geográficas do sistema e o estado dessa informação. Esses métodos são chamados por uma instância de GPSInterface. Eles são:

\begin{itemize}
    \item \texttt{{static} set\_coordinates(): void}
    \item \texttt{{static} set\_gps\_state(): void}
\end{itemize}

\subsubsection{RTCDriver}
Esta classe controla a interface com um módulo \acrshort{rtc}. O método \texttt{update\_rtc(RTCInterface*, time\_t)} é chamado sempre que o módulo \acrshort{rtc} precisa ser atualizado. Recebe como parâmetro um ponteiro para a instância do \texttt{RTCInterface} que será atualizada e a data e hora. O método \texttt{sync\_time\_from\_RTC(RTCInterface*)} retorna a data e hora a partir do ponteiro ao tipo \texttt{RTCInterface} passado como parâmetro

\subsubsection{DataContainer}
Esta é uma classe abstrata que contém informações sobre a leitura de uma variável. Essas informações são: o identificador da variável que está sendo medida e o valor dessa variável; as coordenadas; e a data e hora onde a valor foi medido. Objetos desta classe são usados para armazenar dados no cartão SD e para enviar postagens \textit{HTTP}. O método \texttt{toCSV(Print*)} é um método virtual puro para formatar os dados de uma leitura de variável e armazená-los em um arquivo \acrshort{csv}. Por se tratar de um método virtual, ele deve ser implementado pelas classes filhas de \texttt{DataContainer}. Desta forma cada aplicação pode ter seu próprio formato de armazenamento das informações.

\subsubsection{HardStorage}
Esta classe contém os métodos para leitura e gravação de e para um cartão SD. Para leitura, o método \texttt{open\_file(const char*)} abre o arquivo no qual as operações de leitura/gravação serão executadas. O método recebe o nome do arquivo como parâmetro. Já para a escrita no cartão, o método \texttt{save\_to\_file<T>(DataContainer*, const char*)} grava dados em um arquivo no cartão SD. O nome do arquivo é passado como parâmetro, juntamente com os dados a serem salvos. A função espera um ponteiro para um \texttt{DataContainer}, que na versão atual do \textit{firmware} são objetos do tipo \texttt{SensorData}. O objeto \texttt{SensorData} implementa o método \texttt{toCSV(Print*)}, que recebe um ponteiro para o arquivo e armazena os dados nele.

\subsection{O módulo Sensores}

As classes deste pacote encapsulam a lógica de leitura de cada sensor, considerando as especificações de cada fabricante. Eles fazem uso das interfaces de sensores implementadas no pacote de Interfaces de \textit{Hardware}. Dois fabricantes de sensores foram utilizados no \textit{hardware} dos equipamentos desenvolvidos no contexto deste trabalho: Alphasense e SPEC Sensors. As interfaces dos sensores Alphasense e SPEC diferem na forma como foram implementadas. As saídas dos sensores Alphasense são dois sinais de tensão analógicos. Os sensores SPEC, por outro lado, fornecem os valores de concentração de gás, temperatura e umidade em uma cadeia de caracteres que é enviada através de uma interface \acrshort{uart}. A Figura \ref{fig:fw-libraries-sensors} mostra um diagrama das classes implementadas para este módulo.

\begin{figure}[h]
    \centering
    \caption{Diagramas de classes do módulo Sensors}
    \includegraphics[width=0.90\linewidth]{chapters//2-CLEAN/Figuras/Diagrama-de-classes-Sensors-Package.png}
    \label{fig:fw-libraries-sensors}
\end{figure}

A base para o interfaceamento dos sensores Alphasense é a leitura de duas entradas analógicas do microcontrolador utilizando a função \texttt{analogRead()} do \textit{framework} Arduino. Por esse motivo, a classe base para modelagem dos sensores Alphasense é a classe \texttt{AnalogSensorInterface}. Ela representa uma entrada analógica identificada pelo atributo \texttt{\_inputPin}, e seu método \texttt{read\_mv()} converte o valor digital adquirido pelo conversor analógico-digital do Arduino, em um valor de tensão entre 0 – 5 V. Este método pode receber como parâmetro uma referência a um objeto do tipo \texttt{Smoother}, que por sua vez deve estar associado a um objeto \texttt{Variable}. Assim, são vinculadas as variáveis físicas modeladas no \textit{firmware} com a respectiva interface de \textit{hardware}; neste caso uma entrada analógica.

A classe \texttt{HeatSensor} representa um sensor que precisa de um tempo de aquecimento para funcionar. A lógica que determina a validade das leituras dos sensores é implementada dentro desta classe, levando em consideração um período de aquecimento para cada sensor. Do \texttt{HeatSensor} derivam as classes que representam os sensores Alphasense e SPEC, uma vez que ambos são sensores eletroquímicos amperométricos que requerem um intervalo de aquecimento para garantir que as leituras sejam válidas. A continuação resumem-se as principais propriedades das classes relacionadas ao interfaceamento dos sensores de gases.

\subsubsection{\texttt{AlphaSenseISB}}
Esta classe representa um sensor Alphasense com um circuito de condicionamento do tipo \acrshort{isb}. O sufixo "ISB" indica que o circuito de condicionamento usado é a placa de detecção individual do fabricante do sensor. Esta classe não incorpora nenhum algoritmo de compensação.

\textbf{Atributos:}
\texttt{\_we: AnalogSensorInterface}: Este é um atributo privado que representa a entrada analógica conectada ao eletrodo de trabalho (WE) do sensor

\textbf{Métodos:}
\texttt{readConc(): double}
\texttt{readConc(Smoother\&): double}
Estes são métodos públicos que convertem o valor de tensão lido pelo atributo \texttt{\_we} em um valor de concentração, levando em consideração a sensibilidade do sensor informada pelo fabricante. A referência ao objeto \texttt{Smoother} associa o sensor à variável física correspondente e retorna um valor suave das leituras da variável.

\subsubsection{\texttt{AlphaSenseCompensator}}
Derivado do \texttt{AlphaSenseISB}, representa um sensor Alphasense com um algoritmo de compensação. Os sensores da série Alphasense B4 podem usar diferentes algoritmos de compensação dependendo do gás ao qual são sensíveis. Por isso, cada algoritmo é inerente a cada objeto e não à classe

\textbf{Atributos:}
\texttt{\_ae: AnalogSensorInterface}: Este é um atributo privado que representa a saída do eletrodo auxiliar (AE) do sensor eletroquímico. O valor de saída deste eletrodo é usado nos algoritmos de compensação.

\textbf{Métodos:}
\texttt{(*comp\_algorithm)(uint8, double, double, double, double, double): double}: Este é um ponteiro para a função que implementa o algoritmo de compensação. As funções recebem como parâmetros as variáveis necessárias para o cálculo do algoritmo, dentre eles a temperatura.

\texttt{{virtual} readConc\_Comp(double): double}
\texttt{readConc\_Comp(Smoother\&, double): double}
Estes são métodos públicos que leem os valores de tensão armazenados nos atributos \texttt{\_we} (herdados do \texttt{AlphaSenseISB}) e \texttt{\_ae}. Eles aplicam o algoritmo de compensação correspondente e retornam um valor de concentração. Ambos os métodos recebem como parâmetros a temperatura ambiente e uma referência a um objeto \texttt{Smoother}, como no \texttt{AlphaSenseISB}.

\subsubsection{\texttt{AlphaOXCompensator}}
Este é um caso especial para sensores de ozônio que utilizam um algoritmo de compensação. Os sensores de ozônio medem, na verdade, a soma das concentrações de ozônio e dióxido de nitrogênio, portanto, o valor da concentração de dióxido de nitrogênio é exigido pelo algoritmo de compensação.

\textbf{Atributos:}
\texttt{\_no2: AlphaSenseCompensator*}: Para acessar o sensor de dióxido de nitrogênio, a classe \texttt{AlphaOXCompensator} usa um ponteiro para um objeto \texttt{AlphaSenseCompensator} que representa o sensor de dióxido de nitrogênio.

\textbf{Métodos:}
\texttt{readConc\_Comp(double): double}: Este método lê o valor da concentração do sensor de ozônio e aplica um algoritmo de compensação considerando também a concentração de dióxido de nitrogênio. Para vincular essas leituras a um objeto do tipo \texttt{Variable}, a classe \texttt{AlphaOXCompensator} utiliza o mesmo método \texttt{readConc\_Comp()} herdado da classe \texttt{AlphaSenseCompensator}, que recebe uma referência a um objeto do tipo \texttt{Smoother}.

\subsubsection{Interface com sensores seriais}

A interface com os sensores SPEC é realizada através da classe abstrata \texttt{SerialSensorInterface}. Esta classe fornece métodos para a leitura dos sensores através da porta serial do microcontrolador. A comunicação entre os sensores e o Arduino pode ser implementada através de uma interface \acrshort{uart} ou através de um barramento RS-485. Ambas as interfaces de comunicação são modeladas nas classes \texttt{UARTSensorInterface} e \texttt{RS485SensorInterface}, que derivam de \texttt{SerialSensorInterface}.

A classe \texttt{specDGS\_sensor} funciona como uma camada intermediária entre a interface de \textit{hardware} e as classes do módulo \texttt{Data}. Por representar um sensor eletroquímico que necessita de um período de aquecimento, esta classe também herda da classe \texttt{HeatSensor}. As instâncias de \texttt{specDGS\_sensor} têm a finalidade de ler e analisar as cadeias de caracteres enviadas pelos sensores SPEC, com as medições de temperatura, umidade e concentração de gás. Esta classe também é responsável por validar as medições levando em consideração o tempo de aquecimento dos sensores e possíveis erros na comunicação serial. O método \texttt{readSensor()} lê os valores de concentração, temperatura e umidade e os disponibiliza aos objetos de tipo \texttt{Variable} correspondentes por meio das referências \texttt{Smoother} que recebe como parâmetros.

O atributo \texttt{\_serial} da classe \texttt{specDGS\_sensor} é um ponteiro para um objeto do tipo \texttt{SerialSensorInterface}, que é atribuído durante a construção de cada instância \texttt{specDGS\_sensor}. O ponteiro pode ser um objeto do tipo \texttt{UARTSensorInterface} ou \texttt{RS485SensorInterface}, dependendo apenas da interface de comunicação implementada no \textit{hardware}. Os objetos \texttt{specDGS\_sensor} representam os sensores SPEC, e as instâncias derivadas da classe abstrata \texttt{SerialSensorInterface} representam a interface com esses sensores, que no \textit{hardware} é uma única porta serial.

\subsection{O módulo \texttt{Data}}

A Figura \ref{fig:fw-libraries-data} mostra o diagrama de classes do módulo \texttt{Data}. Como já mencionado, este módulo funciona como uma camada intermediária que prepara e formata as medições obtidas no \textit{hardware} do sensor para seu armazenamento e transmissão. É formado por duas classes principais: \texttt{Variable} e \texttt{SensorData}.

A classe \texttt{SensorData} método prepara os dados para transmissão remota e armazenamento local. Cada objeto do \texttt{SensorData} está associado a um único objeto do tipo \texttt{Variable}, que representa uma variável física com um identificador único. Vale ressaltar que, embora no \textit{firmware} cada variável física seja representada por um único identificador, no \textit{hardware} uma ou mais dessas variáveis podem estar vinculadas a um mesmo transdutor. O número de identificação que representa cada variável física é o que associa cada objeto da classe \texttt{Variable} ao objeto do tipo \texttt{SensorData} correspondente. Este número é armazenado em cada classe nos atributos \texttt{\_sensorID} e \texttt{\_id} como valores inteiros de 32 bits.

\begin{figure}[h]
    \centering
    \caption{Diagrama de classes do pacote \texttt{Data}}
    \includegraphics[width=0.80\linewidth]{chapters//2-CLEAN/Figuras/Diagrama-de-classes-Data-Package.png}
    \label{fig:fw-libraries-data}
\end{figure}

Objetos do tipo \texttt{SensorData} contêm o valor das variáveis físicas às quais estão associados, juntamente com informações sobre a data, hora e local onde as medições foram feitas. O valor de cada variável é armazenado no atributo \texttt{\_value}, o qual pode ser um dado bruto medido em determinado instante de tempo ou uma média de valores adquiridos durante uma janela temporal. O método \texttt{toCSV()} consolida e prepara as informações do valor medido pelo sensor, sua geolocalização, e a data e hora em que a medição foi realizada, no formato \acrshort{csv}.

A classe \texttt{Variable} atua como uma camada intermediária entre a camada de \textit{hardware} do sensor e a classe \texttt{SensorData}. Os objetos desta classe representam as variáveis físicas que estão sendo monitoradas, porém, não contêm suas quantidades, pois esses valores estão armazenados em objetos do tipo \texttt{SensorData}. Como já mencionado, o atributo \texttt{\_id} contém o identificador da variável física que representa. O atributo \texttt{\_unit} representa a unidade de medida da variável física que está sendo monitorada. O atributo \texttt{\_var}, do tipo \texttt{Smoother}, funciona como um buffer de memória no qual os objetos que implementam a interface de \textit{hardware} dos sensores podem colocar as amostras da variável medida. Desse mesmo buffer, o objeto \texttt{SensorData} associado pode extrair o valor médio das amostras. O número de amostras de cada buffer depende da capacidade que for programada. Os diagramas nas Figuras \ref{fig:fw-read-write-smoother} e\ref{fig:fw-sense-seq} ilustram este processo.

\begin{figure}[t]
    \centering
    \caption{Processo de leitura de uma variável}
    \begin{subfigure}{0.495\textwidth}
        \includegraphics[width=\textwidth]{chapters/2-CLEAN/Figuras/Diagrama R_W Buffer Smoother.png}
        \caption{Processo de escrita e leitura no buffer da classe \textit{Smoother}}
        \label{fig:fw-read-write-smoother}
    \end{subfigure}
    \hfill
    \begin{subfigure}{0.495\textwidth}
        \includegraphics[width=\textwidth]{chapters/2-CLEAN/Figuras/Diagramas de sequências.png}
        \caption{Diagrama de sequências do método \textit{sense()}}
        \label{fig:fw-sense-seq}
    \end{subfigure}
    \hfill
    \label{fig:fw-read-variable}
    \fonte{Desenvolvido pelo autor (2023)}
\end{figure}

Os objetos que representam os sensores, gravam as leituras de cada variável física no buffer de amostragem através do método \texttt{smooth()} da classe \texttt{Smoother} (\ref{fig:fw-read-write-smoother}). Já a classe \texttt{Variable} acessa a média das amostras invocando o método \texttt{getAve()} do atributo \texttt{\_var}. Esse valor médio é transferido para o objeto \texttt{SensorData} associado por meio do método \texttt{setValue()}. O processo de leitura e transferência do valor médio das amostras para o objeto \texttt{SensorData} acontece dentro do método \texttt{sense()} definido na classe \texttt{Variable}. A Figura \ref{fig:fw-sense-seq} mostra o diagrama de sequência para este método.

A função \texttt{sense()} é um método virtual puro, portanto, as instâncias que derivam da classe abstrata \texttt{Variable} devem implementá-la. A sequência principal de ações executadas pelo método é comum a todas as classes derivadas. Quando o método \texttt{sense()} é invocado, a classe filha de \texttt{Variable} acessa, através do método \texttt{getAve()} de \texttt{Smoother}, o valor médio das amostras. Este valor é então passado para o objeto do tipo \texttt{SensorData} associado através do método \texttt{setValue()}. O objeto do tipo \texttt{SensorData}, por sua parte, armazena a data, a hora e o local onde foi feita a medição, bem como o status da leitura (método \texttt{getReadSt()}).

Os tipos de variáveis físicas implementadas na versão atual do \textit{firmware} foram: Temperatura, Umidade, Concentração de gás e Entrada Analógica. Esta última representa uma tensão analógica que pode ser lida como um sinal de tensão entre 0 – 5 V. Essas variáveis foram modeladas em classes filhas de \texttt{Variable} como \texttt{Temperature}, \texttt{Humidity}, \texttt{GasConcentration} e \texttt{AnalogInput}. Por serem classes filhas, todas possuem os mesmos atributos de \texttt{Variable}, mas cada uma implementa seu próprio método \texttt{sense()}.

% ----------------------------------------------------------
\section{A API Renovar}
% ----------------------------------------------------------

Renovar é uma plataforma Web que fornece dados de sensores de ar para visualização e uma \acrshort{api} para integração de sensores de ar \acrshort{iot}. Um \acrshort{mvp} da plataforma foi desenvolvido em parceria com o Departamento de Informática e Estatística da UFSC \cite{Teixeira2018} e foi continuado no \acrshort{lcqar} nos anos seguintes. A plataforma é composta por um banco de dados, um serviço de \textit{back-end} – desenvolvido em linguagem Java utilizando Spring Boot –, e uma aplicação \textit{front-end} criada em Angular, com Ionic e TypeScript. Seu acesso é gratuito e aberto para pesquisas e análises ambientais. A Figura \ref{fig:renovar-map} ilustra o painel principal da plataforma, o qual consiste em um mapa que mostra a localização dos dispositivos de monitoramento. A Figura \ref{fig:renovar-series} ilustra o painel de séries temporais, onde o usuário consegue visualizar o histórico de determinada variável num intervalo de tempo selecionável.

\begin{figure}[h]
    \centering
    \caption{Aplicação web Renovar}
    \begin{subfigure}{0.495\textwidth}
        \includegraphics[width=\textwidth]{chapters/2-CLEAN/Figuras/Renovar main panel.jpeg}
        \caption{Painel principal}
        \label{fig:renovar-map}
    \end{subfigure}
    \hfill
    \begin{subfigure}{0.495\textwidth}
        \includegraphics[width=\textwidth]{chapters/2-CLEAN/Figuras/ Renovar time series panel.jpg}
        \caption{Painel de visualização de séries temporais}
        \label{fig:renovar-series}
    \end{subfigure}
    \hfill
    \label{fig:renovar-map-and-series}
    \fonte{Desenvolvido pelo autor (2023)}
\end{figure}

O sistema recebe dados de dispositivos \acrshort{iot}, como concentração de poluentes atmosféricos, temperatura e umidade relativa. Os dados são armazenados em um banco de dados como séries temporais que podem ser visualizadas online na plataforma \textit{web}. O software consiste em (1) um banco de dados MySQL que armazena as leituras do dispositivo e demais dados necessários à plataforma, como usuários, dispositivos cadastrados, poluentes e unidades; (2) um back-end RESTful, desenvolvido em Java utilizando Spring Boot, que é responsável por coletar dados do banco de dados e prepará-los para o frontend; e (3) o front-end, desenvolvido para ser multiplataforma, disponibilizando a interface com o usuário conforme ilustrado nas imagens da Figura \ref{fig:renovar-map-and-series}. O banco de dados, backend e frontend estão hospedados em um servidor da Universidade Federal de Santa Catarina.

\begin{figure}[h]
    \centering
    \caption{Estrutura da aplicação \textit{Web} Renovar}
    \includegraphics[width=0.80\linewidth]{chapters//2-CLEAN/Figuras/Renovar Structure (PT).png}
    \label{fig:renovar-structure}
    \fonte{Desenvolvido pelo autor (2023)}
\end{figure}

A Figura \ref{fig:renovar-structure} ilustra o funcionamento do serviço em geral. Os dispositivos coletam dados ambientais e os enviam para o banco de dados pela internet. O backend recebe solicitações do frontend e coleta os dados necessários do banco de dados. Caso os dados necessitem de algum tratamento (ex.: cálculo de valores médios ou filtragem), o backend executa as operações necessárias e envia as informações processadas de volta ao frontend. O frontend, por outro lado, implementa a interface com o usuário e gera os resultados das operações solicitadas, como visualizar os dados como séries temporais e baixar os dados como arquivo \acrshort{csv}.

\subsection{Banco de dados}

O banco de dados foi construído utilizando \textit{MySQL} como linguagem de consulta e \textit{phpMyAdmin} para administração. A Figura \ref{fig:renovar-database} ilustra a estrutura do banco, que consiste em oito entidades, cada uma com a sua função, atributos e relacionamentos. Elas são as entidades \textit{User}, \textit{Device}, \textit{Sensor}, \textit{Sample}, \textit{Coordinates}, \textit{Pollutant}, \textit{Unit} e \textit{Type}.

\begin{figure}[h]
    \centering
    \caption{Entidades do banco de dados Renovar}
    \includegraphics[width=0.80\linewidth]{chapters//2-CLEAN/Figuras/Renovar Database.png}
    \label{fig:renovar-database}
    \fonte{Desenvolvido pelo autor (2023)}
\end{figure}

A entidade \textit{Device} representa na prática os dispositivos de coleta, mais especificamente os monitores de qualidade do ar. A entidade \textit{Sensor} representa um sensor de determinada variável física. Relaciona-se com a entidade \textit{Device} com uma cardinalidade 1:N, ou seja, um dispositivo pode ter N sensores enquanto um sensor pode pertencer apenas a um dispositivo. \textit{Pollutant} representa as variáveis que são monitoradas no meio ambiente pelos dispositivos \acrshort{iot}. A relação entre \textit{Pollutant} e \textit{Sensor} também está caracterizada por uma cardinalidade 1:N, já que um sensor possui um único poluente, mas um mesmo poluente pode estar associado a N sensores. Cada poluente tem associado também uma unidade de medida e um tipo de unidade. A entidade \textit{Sample} representa uma leitura de determinada variável física; existem N amostras por sensor. Cada amostra tem associada uma coordenada, que é a localização geográfica onde o valor da amostra será mostrado no mapa. Os dispositivos para medição em locais fixos também tem associados uma Coordenada e consequentemente, todas as amostras dos sensores desse dispositivo devem possuir os mesmos valores de latitude e longitude. Por último, a entidade \textit{User} é utilizada para controle de acesso à plataforma. Embora o acesso aos dados de Renovar não precise de uma etapa de login, a escrita ou envio de dados para a plataforma precisa de um cadastro prévio. Assim, cada dispositivo deve ter um usuário associado para conseguir armazenar as leituras no banco de dados de Renovar.

\subsection{A aplicação \textit{Back-end}}

O backend é uma aplicação autônoma construída na linguagem de programação \textit{Java}, com o \textit{framework Spring Boot} e a ferramenta \textit{Maven}. Esta aplicação é responsável por receber e organizar as leituras enviadas pelos dispositivos \acrshort{iot}, no banco de dados e atender as requisições \textit{HTTP} do lado do cliente (\textit{front-end}). Igualmente a aplicação realiza algumas operações básicas de análise de dados como filtragem e agrupação para gráficos tipo \textit{box-plots}.

Quatro camadas compõem a aplicação: os controladores \acrshort{rest}, a camada de serviço, a camada de acesso aos dados e a camada de domínio (Figura \ref{fig:renovar-backend}) \cite{Teixeira2018}. São os controladores os que estabelecem os \textit{endpoints}, mensagens, conteúdos e cabeçalhos de cada recurso do \textit{back-end}. A Camada de Serviço está abaixo dos controladores \acrshort{rest} e funciona como um mediador entre a Camada de Acesso aos Dados e a Camada de Comunicação. É ela também quem define as regras de negócio da aplicação. A Camada de Domínio replica a modelagem do banco de dados em classes que são utilizadas pelos distintos componentes da aplicação; cada classe representa uma tabela. Por último a Camada de Acesso aos Dados tem a responsabilidade de acessar diretamente o banco de dados e disponibilizar a informação para as camadas superiores. A Figura \ref{fig:renovar-backend-dao} ilustra as classes utilizadas na Camada de Acesso aos Dados.

\begin{figure}[h]
    \centering
    \caption{Camadas da aplicação \textit{back-end} Renovar}
    \includegraphics[width=0.80\linewidth]{chapters//2-CLEAN/Figuras/Renovar back-end strcuture.png}
    \label{fig:renovar-backend}
    \fonte{\cite{Teixeira2018}}
\end{figure}

\begin{figure}[h]
    \centering
    \caption{Classes de acesso ao dados da aplicação \textit{back-end} Renovar}
    \includegraphics[width=0.80\linewidth]{chapters//2-CLEAN/Figuras/Classes de acesso aos Dados.png}
    \label{fig:renovar-backend-dao}
    \fonte{Desenvolvido pelo autor (2023)}
\end{figure}

O processo de coleta das medições enviadas pelos dispositivos de monitoramento na aplicação \textit{back-end} consiste, de forma simplificada, no seguinte fluxo de execução. Os controladores \acrshort{rest} processam as requisições de tipo \textit{POST HTTP} enviadas pelos dispositivos \acrshort{iot}, e transferem as informações até a camada de serviço. Ali os dados são validados e instâncias da entidade \textit{Sample} são criadas e transferidas até a camada de acesso aos dados onde as amostras são armazenadas na tabela correspondente do banco de dados. As requisições e os \textit{endpoints} disponibilizados na \acrshort{api} Renovar pelos controladores \acrshort{rest} são detalhados na documentação da \acrshort{api} disponibilizada no Anexo \ref{anex:renovar-api}.

\subsection{A aplicação \textit{Front-end}}

A aplicação \textit{front-end} foi construída usando o \textit{framework Angular}, junto com outras ferramentas como \textit{HighCharts} e \textit{HighStock} para os gráficos e \textit{Leaflet} para mapas. A tela principal mostra um mapa com todos os dispositivos (Figura \ref{fig:renovar-map-2}). A cor dos dispositivos no mapa indica seu estado: verde para ativo (i.e.: o dispositivo está transmitindo novas medidas), vermelho para inativo (o dispositivo não tem enviado novas medidas em um determinado intervalo de tempo). Quando o usuário seleciona um dispositivo no mapa, um \textit{pop-up} é aberto com informações sobre esse e todos os dispositivos que se encontrem nas mesmas coordenadas geográficas, juntamente com informações sobre seus sensores (Figura \ref{fig:renovar-devices}). A partir desse ponto, o usuário pode acessar a página do dispositivo e escolher um poluente para visualizar sua série histórica dentro de um intervalo de datas (Figura \ref{fig:renovar-series-2}). Se o aparelho for portátil, o usuário também pode ver outro mapa que mostra onde cada amostra foi coletada, conforme mostrado na Figura \ref{fig:renovar-portable-map}. Uma seção de análise de dados possibilita visualizar os valores das leituras dos sensores em gráficos de \textit{box-plots}, agrupando as medições por ano mês ou semana, conforme ilustra a Figura \ref{fig:renovar-data-analysis}.

\begin{figure}[h]
    \centering
    \caption{Aplicação \textit{front-end} da plataforma web Renovar}
    \begin{subfigure}{0.495\textwidth}
        \includegraphics[width=\textwidth]{chapters/2-CLEAN/Figuras/Renovar main panel.jpeg}
        \caption{Mapa de dispositivos}
        \label{fig:renovar-map-2}
    \end{subfigure}
    \hfill
    \begin{subfigure}{0.495\textwidth}
        \includegraphics[width=\textwidth]{chapters/2-CLEAN/Figuras/Renovar Device Information.png}
        \caption{Painel de seleção de dispositivos}
        \label{fig:renovar-devices}
    \end{subfigure}
    \hfill
    \label{fig:renovar-map-and-devices}
    \fonte{Desenvolvido pelo autor (2023)}
\end{figure}

\begin{figure}[h]
    \centering
    \caption{Painéis da aplicação \textit{front-end} de Renovar}
    \begin{subfigure}{0.495\textwidth}
        \includegraphics[width=\textwidth]{chapters/2-CLEAN/Figuras/ Renovar time series panel.jpg}
        \caption{Mapa de dispositivos}
        \label{fig:renovar-series-2}
    \end{subfigure}
    \hfill
    \begin{subfigure}{0.495\textwidth}
        \includegraphics[width=\textwidth]{chapters/2-CLEAN/Figuras/Renovar portable map.png}
        \caption{Painel de seleção de dispositivos}
        \label{fig:renovar-portable-map}
    \end{subfigure}
    \hfill
    \label{fig:renovar-series-and-map}
    \fonte{Desenvolvido pelo autor (2023)}
\end{figure}

\begin{figure}[h]
    \centering
    \caption{Painel de análise de dados da aplicação \textit{back-end} Renovar}
    \includegraphics[width=0.80\linewidth]{chapters//2-CLEAN/Figuras/Renovar Data Analysis Panel.png}
    \label{fig:renovar-data-analysis}
    \fonte{Desenvolvido pelo autor (2023)}
\end{figure}

\section{Dispositivos de hardware desenvolvidos}

Dentro do contexto da iniciativa CLEAN foram desenvolvidos dispositivos para medição da qualidade do ar. Dois desses dispositivos foram protótipos para validação da ideia, concebidos para medições em locais fixos e medições móveis. Numa segunda etapa foram desenvolvidos dispositivos mais robustos com placas de circuito impresso e quadros elétricos para instalação em campo. A continuação serão descritos os equipamentos produzidos.

\subsection{Protótipos de monitores de qualidade do ar de baixo custo}

Foram concebidos dois protótipos de baixo custo para medição de poluentes atmosféricos \cite{Campo2020DEPLOYMENTRESULTS}, um para monitoramento fixo e outro para monitoramento móvel. O hardware de ambos os dispositivos, conforme mostrado na Figura \ref{fig:device-structure}, é composto por três blocos principais: 1) transporte de gás, 2) sensoriamento e 3) microcontrolador. O estágio de transporte de gás captura o ar ambiente nos sensores, que produzem um sinal analógico proporcional à concentração do gás. O microcontrolador, que é um Microchip ATMega2560 embarcado em uma plataforma Arduino Mega, captura as respostas dos sensores e as transforma em dados de concentração de gás. O hardware também obtém a hora e o local onde cada medição foi coletada. O microcontrolador armazena essas informações em um cartão micro SD e as transmite para um servidor web hospedado na Superintendência de Tecnologia da Informação e Comunicação da Universidade, rodando o aplicativo Renovar Web. Uma conexão Wi-Fi é estabelecida por um microcontrolador ESP8266 para transmissão de dados. Um relógio em tempo real e um módulo GPS fornecem informações de data, hora e geolocalização, respectivamente.

\begin{figure}
    \centering
    \caption{Estrutura principal dos dispositivos. a) Medidor de gases fixo, e b) medidor móvel}
    \includegraphics[width=1\linewidth]{chapters//2-CLEAN//Figuras/Estrutura geral protótipos.png}
    \label{fig:device-structure}
\end{figure}

A versão fixa dos dispositivos de monitoramento (Figura \ref{fig:device-structure}a) utiliza seis sensores eletroquímicos do fabricante de sensores Alphasense e quatro sensores eletroquímicos do fabricante SPEC Sensors. Para alimentação de energia do dispositivo utiliza-se uma fonte de 12VCC. Este dispositivo não incorpora módulo \textit{GPS} para geolocalização. A versão móvel (Figura \ref{fig:device-structure}b), por outro lado, utiliza apenas quatro sensores eletroquímicos do fabricante SPEC Sensors. O dispositivo é alimentado por um banco de energia de 5VCC através de uma conexão USB. A Figura 3 ilustra ambos protótipos na versão fixa e móvel. Mais detalhes sobre os dispositivos podem ser encontrados no Apêndice \ref{apendix: hw-prototypes}.

\begin{figure}
    \centering
    \caption{Ilustrações das versões (a) fixa e (b) móvel dos dispositivos de monitoramento}
    \includegraphics[width=1\linewidth]{chapters//2-CLEAN/Figuras/Monitoring Prototypes.jpg}
    \label{fig:monitoring-prototypes}
\end{figure}

\subsection{A placa CLEAN Arduino MEGA}

Com base nos resultados obtidos pelos protótipos e nas experiências alcançadas, foi desenvolvida uma versão mais compacta e atualizada para monitoramento fixo. Esta versão foi chamada de \textit{CLEAN Arduino Mega Board} por causa do microcontrolador Arduino Mega que ela usa como processador principal. A composição do hardware é muito semelhante à dos protótipos, mas os módulos foram montados em uma única \textit{PCB}. A Figura \ref{fig:clean-arduino-mega-board} ilustra o projeto da PCB e uma das placas fabricadas. A PCB foi criada no \textit{software} Eagle, e os arquivos do projeto estão disponíveis nos repositórios do LCQAr da UFSC.

\begin{figure}
    \centering
    \caption{A placa CLEAN Arduino Mega: (a) projeto PCB, (b) vista superior da placa, (c) vista inferior da placa.}
    \includegraphics[width=1\linewidth]{chapters//2-CLEAN/Figuras/CLEAN Arduino Mega Board.jpg}
    \label{fig:clean-arduino-mega-board}
\end{figure}

A Tabela \ref{tab:componentes-clean} do apêndice \ref{apendix: clean-arduino-mega-board} mostra os principais componentes de hardware utilizados na placa, que requer uma tensão de alimentação de 12V, 2A através de um conector de alimentação P4. Possui entradas analógicas para 6 placas de sensores Alphasense da série ISB, barramento RS-485 para futuras expansões, três saídas digitais e conectores para alimentação de ventoinhas de 12V e 5V. A placa foi concebida para suportar conexões Wi-Fi e \textit{GPRS} à Internet. Essas conexões não podem ser utilizadas simultaneamente, o que dependerá de cada aplicação. O usuário pode configurar a placa para usar um ou outro e terá que adaptar o firmware do microcontrolador Arduino correspondentemente.

