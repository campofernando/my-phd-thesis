% ----------------------------------------------------------
\section{Calibração por co-localização}
% ----------------------------------------------------------

A calibração por co-localização consiste em instalar os sensores de baixo custo junto a uma estação de monitoramento de referência para fins de calibração. Para cobrir as diversas condições atmosféricas, a sazonalidade, e a gama de valores de concentração dos gases de interesse e dos gases interferentes, a literatura recomenda uma duração de três a seis meses para a execução das rotinas de calibração e dos testes de validação \cite{Spinelle2013ProtocolPollution}.

Na aplicação apresentada neste trabalho, o instrumento de baixo custo desenvolvido foi instalado por cinco meses junto a uma estação de referência localizada em Vila Moema, no município de Tubarão - SC. A estação é uma das três que fazem parte atualmente da rede de monitoramento do estado de Santa Catarina, operadas pela Diamante Geração de Energia Ltda. \cite{IMASC24}. O município de Tubarão é vizinho Capivari de Baixo, onde se encontra o complexo termelétrico Jorge Lacerda, operado também pela Diamante Geração de Energia Ltda. A Figura \ref{fig:stations-map} ilustra a localização geográfica das estações de monitoramento na região de Capivari de Baixo, e Tubarão. Na Tabela relacionam-se os equipamentos presentes na estação Vila Moema onde foi instalado o instrumento de baixo custo desenvolvido.

\begin{figure}[h]
    \centering
    \caption{Mapa das estações de monitoramento em Tubarão e Capivari de Baixo}
    \includegraphics[width=0.9\textwidth]{chapters/3-ANÁLISE DOS DADOS/Figuras/mapa-de-estações.png}
    \label{fig:stations-map}
    \fonte{\cite{IMASC24}}
\end{figure}

\begin{table}[!ht]
    \centering
    \caption{Relação de equipamentos presentes na estação de monitoramento de referência no município de Tubarão - SC}
    \begin{tabular}{l|l|l|l}
        \textbf{Equipamento} & \textbf{Poluentes} & \textbf{Princípio de operação} & \textbf{Fabricante} \\
        \hline
        Monitor APNA-370 & NO, NO2 e NOx & Quimioluminescência & Horiba \\
        Monitor APOA-370 & O3 & Adsorção ultravioleta & Horiba \\
        Monitor APSA-370 & SO2 & Fluorescência ultravioleta & Horiba \\
        Monitor APMA-370 & CO & Modulação cruzada & Horiba \\
         & & infravermelha sem dispersão & \\
        Monitor BAM 1020 & MP2.5, MP10 & Atenuação de raios beta & Met One \\
        \hline
    \end{tabular}
    \label{tab:reference-equipments}
\end{table}

Os dados registrados pelos equipamentos de referência são compilados em relatórios diários onde as leituras de concentração são registradas como médias horárias. Para a calibração do monitor de baixo custo, o Instituto do Meio Ambiente de Santa Catarina disponibilizou os relatórios de concentrações de poluentes no período e questão. O instrumento de baixo custo desenvolvido registra as leituras dos sensores com um período de amostragem de 15 minutos. Os dados adquiridos foram filtrados, pré-processados e re-amostrados em médias horárias para efetuar a calibração com a estação de referência.

\begin{figure}[h]
    \centering
    \caption{Fluxos de calibração dos sensores de baixo custo}
    \begin{subfigure}{0.9\textwidth}
        \includegraphics[width=\textwidth]{chapters/3-ANÁLISE DOS DADOS/Figuras/Processo de calibração unisensor.png}
        \caption{Fluxo de calibração dos sensores individualmente}
        \label{fig:calibration-unisensor}
    \end{subfigure}
    \begin{subfigure}{0.9\textwidth}
        \includegraphics[width=\textwidth]{chapters/3-ANÁLISE DOS DADOS/Figuras/Processo de calibração multisensor.png}
        \caption{Fluxo de calibração das leituras do equipamento de baixo custo como um todo}
        \label{fig:calibration-multisensor}
    \end{subfigure}
\end{figure}

As Figuras \ref{fig:calibration-unisensor} e \ref{fig:calibration-multisensor} ilustram o processo de preparação, pré-processamento e calibração dos dados dos sensores. As leituras dos sensores de baixo custo e dos instrumentos de referência foram disponibilizadas em arquivos \textit{csv}, que foram utilizados como entradas na etapa de pré-processamento. Para a calibração das leituras do equipamento de baixo custo foram desenvolvidas duas metodologias de calibração. A primeira delas (Figura \ref{fig:calibration-unisensor}) considerou como variáveis de entrada apenas os dados dos sensores específicos para cada poluente e a temperatura. Já a segunda metodologia (Figura \ref{fig:calibration-multisensor}) considerou as leituras de todos os sensores no equipamento para inferir o valor da concentração real de cada poluente.

Para a calibração dos sensores foram explorados diferentes modelos de regressão multivariados, i.e.: o Perceptron Multicamadas (MLP), Regressão Linear Multivariada (MLR), K Vizinhos mais Próximos (KNN) e as Florestas Aleatórias (RF). A escolha desses modelos baseou-se na capacidade de lidar com múltiplas variáveis simultaneamente, proporcionando uma adaptação mais eficaz às complexidades das relações entre os dados dos sensores e as condições ambientais de temperatura. Para encontrar o modelo que melhor explicasse os dados foram realizadas buscas em \textit{grid} que combinaram diferentes parâmetros e variáveis de entrada. Os modelos são avaliados utilizando validações cruzadas com k = 3 considerando métricas essenciais, i.e.: o coeficiente de determinação (r2), o erro médio quadrático (RMSE) e o erro absoluto médio (MAE). Também foram avaliados os modelos segundo suas complexidades utilizando os coeficientes \acrshort{aic} e \acrshort{bic}.