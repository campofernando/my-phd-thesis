Uma vez pré-processados os dados dos sensores procedeu-se a realizar a calibração das leituras de \acrshort{co}, \acrshort{o3} e \acrshort{mp}, utilizando como referência os dados provenientes de uma estação de monitoramento de qualidade do ar. Foram explorados diferentes modelos multivariados para a calibração, i.e.: o Perceptron Multicamadas (MLP), Regressão Linear Multivariada (MLR), K Vizinhos mais Próximos (KNN) e as Florestas Aleatórias (RF). A escolha desses modelos baseou-se na capacidade de lidar com múltiplas variáveis simultaneamente, proporcionando uma adaptação mais eficaz às complexidades das relações entre os dados dos sensores e as condições ambientais de temperatura.

A calibração foi realizada em duas fases. Inicialmente, considerou-se apenas os dados do sensor em questão e a temperatura como variáveis de entrada. Para encontrar o modelo que melhor explicasse os dados foi realizada uma busca em \textit{grid} que combina diferentes parâmetros e variáveis de entrada e avalia cada modelo utilizando validações cruzadas com k = 3 Posteriormente, expandiu-se o escopo para incluir os dados de outros sensores disponíveis. Essa abordagem permitiu avaliar o impacto da inclusão de mais variáveis na precisão da calibração. A avaliação comparativa dos modelos de calibração foi realizada mediante métricas essenciais, incluindo o coeficiente de determinação (r2), o erro médio quadrático (RMSE) e o erro absoluto médio (MAE). A utilização de validações cruzadas proporciona uma análise mais robusta e imparcial do desempenho dos modelos.

A expectativa é que a implementação desses modelos de calibração aprimore significativamente a qualidade dos dados, fornecendo uma base sólida para análises subsequentes relacionadas à qualidade do ar. Os resultados desta pesquisa têm o potencial não apenas de beneficiar a precisão dos dados locais, mas também de contribuir para o desenvolvimento de metodologias aprimoradas de calibração em ambientes de monitoramento ambiental.

\subsection{Calibração das leituras do sensor CO-B4}

Nas Figuras \ref{fig:data-co-reference-time-series} e \ref{fig:data-co-reference-corr} apresentam-se as leituras de \acrshort{co} obtidas pelo sensor CO-B4 de Alphasense e a estação de referência. Observa-se que as leituras do sensor CO-B4 em geral subestimaram os valores de concentração de referência. Os testes de Spearman e Kendall revelaram a existência de correlação entre as medições com o sensor de baixo custo e a referência com coeficientes de 0.3 e 0.2 respectivamente.

\begin{figure}[h]
    \centering
    \caption{Séries temporais e gráficos de dispersão das medições de \acrshort{co}}
    \begin{subfigure}{0.495\textwidth}
        \includegraphics[width=\textwidth]{chapters/3-RESULTADOS CAMPO/Figuras/co-b4-reference-time-series.png}
        \caption{Séries temporais das leituras do sensor CO-B4 e a estação de referência}
        \label{fig:data-co-reference-time-series}
    \end{subfigure}
    \hfill
    \begin{subfigure}{0.495\textwidth}
        \includegraphics[width=\textwidth]{chapters/3-RESULTADOS CAMPO/Figuras/co-b4-reference-correlation.png}
        \caption{Gráfico de dispersão das leituras do sensor CO-B4 e a estação de referência}
        \label{fig:data-co-reference-corr}
    \end{subfigure}
    \fonte{Desenvolvido pelo autor (2023)}
\end{figure}

\subsubsection{Calibração utilizando as leituras do sensor CO-B4 e a temperatura}

Na Tabela \ref{tab:data-co-br-calib-results} resumem-se os melhores modelos encontrados pela busca em \textit{grid} para calibrar as leituras do sensor CO-B4. A Figura \ref{fig:data-co-b4-models-performance} apresenta o desempenho dos modelos e as variáveis de entrada considerando os valores de r2, RMSE e MAE.

\begin{table}[h]
    \caption{Resultados da calibração do sensor CO-B4}
    \centering
    \begin{tabularx}{0.95\textwidth}[h]{
         >{\raggedright\hsize=.08\hsize\arraybackslash}X
         >{\raggedright\hsize=.8\hsize\arraybackslash}X 
         >{\raggedright\hsize=.5\hsize\arraybackslash}X
         >{\raggedright\hsize=.5\hsize\arraybackslash}X 
         >{\raggedright\hsize=.5\hsize\arraybackslash}X }
        \hline
        Var. & Modelo & R2 & RMSE & MAE\\ [0.5ex]
        \hline
        CO & \textbf{MLP}: & -0.64 ± 0.49 & -0.07 ± 0.01 & -0.05 \\ [0.5ex]
           & alpha = 0.01 &  & & \\ [0.5ex]
           & hidden layers = (200, 10) & & & \\ [0.5ex]
           & & & & \\ [0.5ex]
           & \textbf{MLR} & -0.61 ± 0.47 & -0.07 ± 0.01 & -0.05 \\ [0.5ex]
           & & & & \\ [0.5ex]
           & \textbf{KNN:} & -0.47 ± 0.29 & -0.06 ± 0.01 & -0.05 \\ [0.5ex]
           & n\_neighbors = 17 & & & \\ [0.5ex]
           & weights = uniform & & & \\ [0.5ex]
           & & & & \\ [0.5ex]
           & \textbf{RF:} & -0.60 ± 0.34 & -0.07 ± 0.01 & -0.05 \\ [0.5ex]
           & max\_depth = 30 & & & \\ [0.5ex]
           & min\_samples\_leaf = 4 & & & \\ [0.5ex]
           & min\_samples\_split = 10 & & & \\ [0.5ex]
           & n\_estimators = 150 & & & \\ [0.5ex]
        \hline
        CO, T & \textbf{MLP:} & -0.59 ± 0.42 & -0.07 ± 0.01 & -0.05 \\ [0.5ex]
              & alpha = 0.1 & & & \\ [0.5ex]
              & hidden layer = (200, 50) & & & \\ [0.5ex]
              & & & & \\ [0.5ex]
              & \textbf{MLR:} & -0.65 ± 0.45 & -0.07 ± 0.01 & -0.05 \\ [0.5ex]
              & & & & \\ [0.5ex]
              & \textbf{KNN:} & -0.61 ± 0.46 & -0.07 ± 0.01 & -0.05 \\ [0.5ex]
              & n\_neighbors = 15 & & & \\ [0.5ex]
              & weights = uniform & & & \\ [0.5ex]
              & & & & \\ [0.5ex]
              & \textbf{RF:} & -0.72 +/- 0.50 & -0.07 ± 0.01 & -0.05 \\ [0.5ex]
              & max\_depth = None & & & \\ [0.5ex]
              & min\_samples\_leaf = 4 & & & \\ [0.5ex]
              & min\_samples\_split = 2 & & & \\ [0.5ex]
              & n\_estimators = 100 & & & \\ [0.5ex]
        \hline
    \end{tabularx}
    \label{tab:data-co-br-calib-results}
    \fonte{Desenvolvido pelo autor}
\end{table}

\begin{figure}[h]
    \centering
    \caption{Resultados dos modelos de calibração aplicados as leituras do sensor CO-B4}
    \includegraphics[width=\textwidth]{chapters/3-RESULTADOS CAMPO/Figuras/co-b4-models-performance.png}
    \label{fig:data-co-b4-models-performance}
    \fonte{Desenvolvido pelo autor (2023)}
\end{figure}

\subsection{Calibração das leituras dos sensores OX-B431}

Nas Figuras \ref{fig:data-o3-reference-time-series}, \ref{fig:data-o3-1-reference-corr} e \ref{fig:data-o3-2-reference-corr} apresentam-se as leituras de \acrshort{o3} obtidas pelos sensores OX-B431 de Alphasense e a estação de referência. Observa-se que as leituras do sensor OX-B431 em geral superestimaram os valores de concentração de referência. Os testes de Spearman e Kendall revelaram a existência de correlação entre as medições com o sensor de baixo custo e a referência com coeficientes de 0.3 e 0.2 respectivamente.

\begin{figure}[h]
    \centering
    \caption{Séries temporais e gráficos de dispersão das medições de \acrshort{o3}}
    \begin{subfigure}{0.99\textwidth}
        \includegraphics[width=\textwidth]{chapters/3-RESULTADOS CAMPO/Figuras/o3-b4-reference-time-series.png}
        \caption{Séries temporais das leituras dos sensores OX-B431 e a estação de referência}
        \label{fig:data-o3-reference-time-series}
    \end{subfigure}
    \hfill
    \begin{subfigure}{0.495\textwidth}
        \includegraphics[width=\textwidth]{chapters/3-RESULTADOS CAMPO/Figuras/o3-b4-1-reference-correlation.png}
        \caption{Gráfico de dispersão das leituras do sensor 1 OX-B431 e a estação de referência}
        \label{fig:data-o3-1-reference-corr}
    \end{subfigure}
    \hfill
    \begin{subfigure}{0.495\textwidth}
        \includegraphics[width=\textwidth]{chapters/3-RESULTADOS CAMPO/Figuras/o3-b4-2-reference-correlation.png}
        \caption{Gráfico de dispersão das leituras do sensor 2 OX-B431 e a estação de referência}
        \label{fig:data-o3-2-reference-corr}
    \end{subfigure}
    \fonte{Desenvolvido pelo autor (2023)}
\end{figure}

\subsubsection{Calibração utilizando as leituras dos sensores OX-B431 e a temperatura}

Na Tabela \ref{tab:data-o3-b4-calib-results} resumem-se os melhores modelos encontrados pela busca em \textit{grid} para calibrar as leituras do sensor OX-B431. A Figura \ref{fig:data-o3-b4-models-performance} apresenta o desempenho dos modelos e as variáveis de entrada considerando os valores de r2, RMSE e MAE.

\begin{table}[h]
    \caption{Resultados da calibração do sensor 1 OX-B431}
    \centering
    \begin{tabularx}{0.95\textwidth}[h]{
         >{\raggedright\hsize=.08\hsize\arraybackslash}X
         >{\raggedright\hsize=.8\hsize\arraybackslash}X 
         >{\raggedright\hsize=.5\hsize\arraybackslash}X
         >{\raggedright\hsize=.5\hsize\arraybackslash}X 
         >{\raggedright\hsize=.5\hsize\arraybackslash}X }
        \hline
        Var. & Modelo & R2 & RMSE & MAE\\ [0.5ex]
        \hline
        \acrshort{o3} (1) & \textbf{MLP}: & -0.38 ± 0.42 & -17.38 ± 3.49 & -13.72 ± 2.55 \\ [0.5ex]
           & alpha = 0.0001 &  & & \\ [0.5ex]
           & hidden layers = (100,) & & & \\ [0.5ex]
           & & & & \\ [0.5ex]
           & \textbf{MLR} & -0.35 ± 0.38 & -17.26 ± 3.57 & -13.63 ± 2.58 \\ [0.5ex]
           & & & & \\ [0.5ex]
           & \textbf{KNN:} & -0.27 ± 0.31 & -17.24 ± 5.05 & -12.91 ± 3.92 \\ [0.5ex]
           & n\_neighbors = 13 & & & \\ [0.5ex]
           & weights = uniform & & & \\ [0.5ex]
           & & & & \\ [0.5ex]
           & \textbf{RF:} & -0.68 ± 0.73 & -18.65 ± 3.27 & -14.87 ± 2.54 \\ [0.5ex]
           & max\_depth = 20 & & & \\ [0.5ex]
           & min\_samples\_leaf = 4 & & & \\ [0.5ex]
           & min\_samples\_split = 10 & & & \\ [0.5ex]
           & n\_estimators = 50 & & & \\ [0.5ex]
        \hline
        \acrshort{o3} (1), T & \textbf{MLP:} & 0.42 ± 0.15 & -11.24 ± 1.13 & -8.62 ± 1.07 \\ [0.5ex]
            & alpha = 0.0001 & & & \\ [0.5ex]
            & hidden layer = (200, 10) & & & \\ [0.5ex]
            & & & & \\ [0.5ex]
            & \textbf{MLR:} & 0.36 ± 0.18 & -11.73 ± 1.23 & -9.12 ± 1.12 \\ [0.5ex]
            & & & & \\ [0.5ex]
            & \textbf{KNN:} & 0.21 ± 0.20 & -13.16 ± 1.93 & -9.96 ± 1.76 \\ [0.5ex]
            & n\_neighbors = 7 & & & \\ [0.5ex]
            & weights = uniform & & & \\ [0.5ex]
            & & & & \\ [0.5ex]
            & \textbf{RF:} & 0.33 ± 0.19 & -11.99 ± 1.23 & -9.24 ± 1.12 \\ [0.5ex]
            & max\_depth = 10 & & & \\ [0.5ex]
            & min\_samples\_leaf = 4 & & & \\ [0.5ex]
            & min\_samples\_split = 10 & & & \\ [0.5ex]
            & n\_estimators = 100 & & & \\ [0.5ex]
        \hline
    \end{tabularx}
    \label{tab:data-o3-1-b4-calib-results}
    \fonte{Desenvolvido pelo autor}
\end{table}

\begin{table}[h]
    \caption{Resultados da calibração do sensor 2 OX-B431}
    \centering
    \begin{tabularx}{0.95\textwidth}[h]{
         >{\raggedright\hsize=.08\hsize\arraybackslash}X
         >{\raggedright\hsize=.8\hsize\arraybackslash}X 
         >{\raggedright\hsize=.5\hsize\arraybackslash}X
         >{\raggedright\hsize=.5\hsize\arraybackslash}X 
         >{\raggedright\hsize=.5\hsize\arraybackslash}X }
        \hline
        Var. & Modelo & R2 & RMSE & MAE\\ [0.5ex]
        \hline
        \acrshort{o3} (2) & \textbf{MLP}: & 0.16 ± 0.13 & -13.72 ± 2.11 & -10.87 ± 1.66 \\ [0.5ex]
           & alpha = 1 &  & & \\ [0.5ex]
           & hidden layers = (50, 50) & & & \\ [0.5ex]
           & & & & \\ [0.5ex]
           & \textbf{MLR} & 0.09 ± 0.14 & -14.34 ± 2.50 & -11.35 ± 2.13 \\ [0.5ex]
           & & & & \\ [0.5ex]
           & \textbf{KNN:} & 0.03 ± 0.35 & -14.27 +/- 1.49 & -11.27 ± 1.39 \\ [0.5ex]
           & n\_neighbors = 20 & & & \\ [0.5ex]
           & weights = uniform & & & \\ [0.5ex]
           & & & & \\ [0.5ex]
           & \textbf{RF:} & -0.03 ± 0.37 & -14.76 ± 1.47 & -11.55 ± 1.34 \\ [0.5ex]
           & max\_depth = 10 & & & \\ [0.5ex]
           & min\_samples\_leaf = 4 & & & \\ [0.5ex]
           & min\_samples\_split = 10 & & & \\ [0.5ex]
           & n\_estimators = 50 & & & \\ [0.5ex]
        \hline
        \acrshort{o3} (2), T & \textbf{MLP:} & 0.23 ± 0.17 & -13.16 ± 2.94 & -10.14 ± 2.62 \\ [0.5ex]
            & alpha = 10 & & & \\ [0.5ex]
            & hidden layer = (10, 10) & & & \\ [0.5ex]
            & & & & \\ [0.5ex]
            & \textbf{MLR:} & 0.38 ± 0.22 & -11.44 ± 0.72 & -8.86 ± 0.86 \\ [0.5ex]
            & & & & \\ [0.5ex]
            & \textbf{KNN:} & 0.19 ± 0.34 & -12.91 ± 0.81 & -9.89 ± 0.92 \\ [0.5ex]
            & n\_neighbors = 2 & & & \\ [0.5ex]
            & weights = uniform & & & \\ [0.5ex]
            & & & & \\ [0.5ex]
            & \textbf{RF:} & 0.28 ± 0.22 & -12.44 ± 1.11 & -9.54 ± 1.08 \\ [0.5ex]
            & max\_depth = 10 & & & \\ [0.5ex]
            & min\_samples\_leaf = 4 & & & \\ [0.5ex]
            & min\_samples\_split = 10 & & & \\ [0.5ex]
            & n\_estimators = 150 & & & \\ [0.5ex]
        \hline
    \end{tabularx}
    \label{tab:data-o3-b4-calib-results}
    \fonte{Desenvolvido pelo autor}
\end{table}

\begin{table}[h]
    \caption{Resultados da calibração das leituras de \acrshort{o3} utilizando os sensores 1 e 2 OX-B431}
    \centering
    \begin{tabularx}{0.95\textwidth}[h]{
         >{\raggedright\hsize=.08\hsize\arraybackslash}X
         >{\raggedright\hsize=.8\hsize\arraybackslash}X 
         >{\raggedright\hsize=.5\hsize\arraybackslash}X
         >{\raggedright\hsize=.5\hsize\arraybackslash}X 
         >{\raggedright\hsize=.5\hsize\arraybackslash}X }
        \hline
        Var. & Modelo & R2 & RMSE & MAE\\ [0.5ex]
        \hline
        \acrshort{o3} (1), \acrshort{o3} (2) & \textbf{MLP}: & 0.24 ± 0.15 & -13.01 ± 1.82 & -10.02 ± 1.55 \\ [0.5ex]
           & alpha = 1 &  & & \\ [0.5ex]
           & hidden layers = (200, 4) & & & \\ [0.5ex]
           & & & & \\ [0.5ex]
           & \textbf{MLR} & 0.14 ± 0.16 & -13.85 ± 2.26 & -10.81 ± 2.02 \\ [0.5ex]
           & & & & \\ [0.5ex]
           & \textbf{KNN:} & 0.12 ± 0.24 & -13.90 ± 1.86 & -10.86 ± 1.72 \\ [0.5ex]
           & n\_neighbors = 20 & & & \\ [0.5ex]
           & weights = uniform & & & \\ [0.5ex]
           & & & & \\ [0.5ex]
           & \textbf{RF:} & 0.13 ± 0.24 & -13.70 ± 1.54 & -10.64 ± 1.43 \\ [0.5ex]
           & max\_depth = 10 & & & \\ [0.5ex]
           & min\_samples\_leaf = 2 & & & \\ [0.5ex]
           & min\_samples\_split = 10 & & & \\ [0.5ex]
           & n\_estimators = 100 & & & \\ [0.5ex]
        \hline
        \acrshort{o3} (1), \acrshort{o3} (2), T & \textbf{MLP:} & 0.29 ± 0.21 & -12.37 ± 1.59 & -8.89 ± 0.82 \\ [0.5ex]
            & alpha = 0.001 & & & \\ [0.5ex]
            & hidden layer = (200,) & & & \\ [0.5ex]
            & & & & \\ [0.5ex]
            & \textbf{MLR:} & 0.39 ± 0.21 & -11.37 ± 0.82 & -8.82 ± 0.88 \\ [0.5ex]
            & & & & \\ [0.5ex]
            & \textbf{KNN:} & 0.19 ± 0.34 & -12.97 ± 1.18 & -9.92 ± 1.37 \\ [0.5ex]
            & n\_neighbors = 13 & & & \\ [0.5ex]
            & weights = distance & & & \\ [0.5ex]
            & & & & \\ [0.5ex]
            & \textbf{RF:} & 0.30 ± 0.19 & -12.34 ± 1.61 & -9.57 ± 1.58 \\ [0.5ex]
            & max\_depth = 10 & & & \\ [0.5ex]
            & min\_samples\_leaf = 1 & & & \\ [0.5ex]
            & min\_samples\_split = 10 & & & \\ [0.5ex]
            & n\_estimators = 150 & & & \\ [0.5ex]
        \hline
    \end{tabularx}
    \label{tab:data-o3-1-2-b4-calib-results}
    \fonte{Desenvolvido pelo autor}
\end{table}

\begin{figure}[h]
    \centering
    \caption{Resultados dos modelos de calibração aplicados às leituras dos sensores OX-B431}
    \includegraphics[width=\textwidth]{chapters/3-RESULTADOS CAMPO/Figuras/o3-b4-models-performance.png}
    \label{fig:data-o3-b4-models-performance}
    \fonte{Desenvolvido pelo autor (2023)}
\end{figure}

\subsection{Calibração das leituras de \acrshort{mp10} do sensor OPC-N3}

Nas Figuras \ref{fig:data-pm10-reference-time-series} e \ref{fig:data-pm10-reference-corr} apresentam-se as leituras de \acrshort{mp10} obtidas pelo sensor OPC-N3 de Alphasense e a estação de referência. Observa-se que as leituras do sensor OPC-N3 em geral subestimaram os valores de concentração de referência. Os testes de Spearman e Kendall revelaram a existência de correlação entre as medições com o sensor de baixo custo e a referência com coeficientes de 0.3 e 0.2 respectivamente.

\begin{figure}[h]
    \centering
    \caption{Séries temporais e gráficos de dispersão das medições de \acrshort{mp10}}
    \begin{subfigure}{0.495\textwidth}
        \includegraphics[width=\textwidth]{chapters/3-RESULTADOS CAMPO/Figuras/pm10-reference-time-series.png}
        \caption{Séries temporais das leituras de \acrshort{mp10} do sensor OPC-N3 e a estação de referência}
        \label{fig:data-pm10-reference-time-series}
    \end{subfigure}
    \hfill
    \begin{subfigure}{0.495\textwidth}
        \includegraphics[width=\textwidth]{chapters/3-RESULTADOS CAMPO/Figuras/pm10-reference-correlation.png}
        \caption{Gráfico de dispersão das leituras de \acrshort{mp10} do sensor OPC-N3 e a estação de referência}
        \label{fig:data-pm10-reference-corr}
    \end{subfigure}
    \fonte{Desenvolvido pelo autor (2023)}
\end{figure}

\subsubsection{Calibração utilizando as leituras de \acrshort{mp10}do sensor OPC-N3 e a temperatura}

Na Tabela \ref{tab:data-pm10-calib-results} resumem-se os melhores modelos encontrados pela busca em \textit{grid} para calibrar as leituras de \acrshort{mp10} do sensor OPC-N3. A Figura \ref{fig:data-pm10-models-performance} apresenta o desempenho dos modelos e as variáveis de entrada considerando os valores de r2, RMSE e MAE.

\begin{table}[h]
    \caption{Resultados da calibração das leituras de \acrshort{mp10} do sensor OPC-N3}
    \centering
    \begin{tabularx}{0.95\textwidth}[h]{
         >{\raggedright\hsize=.15\hsize\arraybackslash}X
         >{\raggedright\hsize=.8\hsize\arraybackslash}X 
         >{\raggedright\hsize=.5\hsize\arraybackslash}X
         >{\raggedright\hsize=.5\hsize\arraybackslash}X 
         >{\raggedright\hsize=.5\hsize\arraybackslash}X }
        \hline
        Var. & Modelo & R2 & RMSE & MAE\\ [0.5ex]
        \hline
        \acrshort{mp10} & \textbf{MLP}: & -0.05 ± 0.036 & -9.77 ± 0.75 & -7.41 ± 0.49 \\ [0.5ex]
           & alpha = 10 &  & & \\ [0.5ex]
           & hidden layers = (10, 10) & & & \\ [0.5ex]
           & & & & \\ [0.5ex]
           & \textbf{MLR} & -0.01 ± 0.03 & -9.57 ± 0.98 & -7.26 ± 0.62 \\ [0.5ex]
           & & & & \\ [0.5ex]
           & \textbf{KNN:} & -0.14 ± 0.07 & -10.18 ± 0.62 & -7.71 ± 0.45 \\ [0.5ex]
           & n\_neighbors = 20 & & & \\ [0.5ex]
           & weights = uniform & & & \\ [0.5ex]
           & & & & \\ [0.5ex]
           & \textbf{RF:} & -0.19 ± 0.11 & -10.33 ± 0.52 & -7.80 ± 0.40 \\ [0.5ex]
           & max\_depth = 10 & & & \\ [0.5ex]
           & min\_samples\_leaf = 2 & & & \\ [0.5ex]
           & min\_samples\_split = 10 & & & \\ [0.5ex]
           & n\_estimators = 100 & & & \\ [0.5ex]
        \hline
        \acrshort{mp10}, T & \textbf{MLP:} & -0.29 ± 0.27 & -10.68 ± 0.96 & -8.33 ± 0.78 \\ [0.5ex]
              & alpha = 1 & & & \\ [0.5ex]
              & hidden layer = (4, 50) & & & \\ [0.5ex]
              & & & & \\ [0.5ex]
              & \textbf{MLR:} & 0.10 ± 0.08 & -9.04 ± 1.12 & -6.73 ± 0.69 \\ [0.5ex]
              & & & & \\ [0.5ex]
              & \textbf{KNN:} & -0.02 ± 0.09 & -9.57 ± 0.46 & -7.29 ± 0.27 \\ [0.5ex]
              & n\_neighbors = 20 & & & \\ [0.5ex]
              & weights = uniform & & & \\ [0.5ex]
              & & & & \\ [0.5ex]
              & \textbf{RF:} & -0.17 ± 0.27 & -10.12 ± 0.32 & -7.89 ± 0.28 \\ [0.5ex]
              & max\_depth = 10 & & & \\ [0.5ex]
              & min\_samples\_leaf = 4 & & & \\ [0.5ex]
              & min\_samples\_split = 10 & & & \\ [0.5ex]
              & n\_estimators = 150 & & & \\ [0.5ex]
        \hline
    \end{tabularx}
    \label{tab:data-pm10-calib-results}
    \fonte{Desenvolvido pelo autor}
\end{table}

\begin{figure}[h]
    \centering
    \caption{Resultados dos modelos de calibração aplicados as leituras de \acrshort{mp10} do sensor OPC-N3}
    \includegraphics[width=\textwidth]{chapters/3-RESULTADOS CAMPO/Figuras/pm10-models-performance.png}
    \label{fig:data-pm10-models-performance}
    \fonte{Desenvolvido pelo autor (2023)}
\end{figure}

\subsection{Calibração das leituras do equipamento de medição utilizando vários sensores por poluente}

\subsubsection{Calibração das leituras de \acrshort{co}}

A Figura \ref{fig:data-co-all-models-performance} apresenta os valores de R2 dos 10 melhores modelos de calibração calculados para as leituras de \acrshort{co}. Observa-se que apesar do valor médio de R2 obtido nas validações cruzadas continuar sendo negativo, obtiveram-se máximos de aproximadamente 0.1 para alguns conjuntos de dados ao utilizar regressões com k vizinhos mais próximos considerando as leituras de \acrshort{o3} e \acrshort{mp10}.

\begin{figure}[h]
    \centering
    \caption{Resultados dos 10 melhores modelos de calibração aplicados as leituras de \acrshort{co} do sensor CO-B4}
    \includegraphics[width=\textwidth]{chapters/3-RESULTADOS CAMPO/Figuras/co-all-models-performance.png}
    \label{fig:data-co-all-models-performance}
    \fonte{Desenvolvido pelo autor (2023)}
\end{figure}

A Figura \ref{fig:data-co-all-models-comparison} mostra uma comparação do desempenho, em termos do valor médio de R2, de cada modelo de regressão aplicado para diferentes combinações de variáveis de entrada.

\begin{figure}[h]
    \centering
    \caption{Comparação dos modelos de calibração aplicados as leituras de \acrshort{co} do sensor CO-B4}
    \includegraphics[width=\textwidth]{chapters/3-RESULTADOS CAMPO/Figuras/co-all-models-comparison.png}
    \label{fig:data-co-all-models-comparison}
    \fonte{Desenvolvido pelo autor (2023)}
\end{figure}

\subsubsection{Calibração das leituras de \acrshort{o3}}

A Figura \ref{fig:data-o3-all-models-performance} apresenta os valores de R2 dos 10 melhores modelos de calibração calculados para as leituras de \acrshort{o3}. Observa-se que apesar do valor médio de R2 obtido nas validações cruzadas continuar sendo negativo, obtiveram-se máximos de aproximadamente 0.1 para alguns conjuntos de dados ao utilizar regressões com k vizinhos mais próximos considerando as leituras de \acrshort{co} e \acrshort{mp10}.

\begin{figure}[h]
    \centering
    \caption{Resultados dos 10 melhores modelos de calibração aplicados as leituras de \acrshort{o3} do sensor OX-B431}
    \includegraphics[width=\textwidth]{chapters/3-RESULTADOS CAMPO/Figuras/o3-all-models-performance.png}
    \label{fig:data-o3-all-models-performance}
    \fonte{Desenvolvido pelo autor (2023)}
\end{figure}

A Figura \ref{fig:data-o3-all-models-comparison} mostra uma comparação do desempenho, em termos do valor médio de R2, de cada modelo de regressão aplicado para diferentes combinações de variáveis de entrada.

\begin{figure}[h]
    \centering
    \caption{Comparação dos modelos de calibração aplicados as leituras de \acrshort{o3} do sensor OX-B431}
    \includegraphics[width=\textwidth]{chapters/3-RESULTADOS CAMPO/Figuras/o3-all-models-comparison.png}
    \label{fig:data-o3-all-models-comparison}
    \fonte{Desenvolvido pelo autor (2023)}
\end{figure}

\subsubsection{Calibração das leituras de \acrshort{mp10}}

A Figura \ref{fig:data-pm10-all-models-performance} apresenta os valores de R2 dos 10 melhores modelos de calibração calculados para as leituras de \acrshort{mp10}. Observa-se que apesar do valor médio de R2 obtido nas validações cruzadas continuar sendo negativo, obtiveram-se máximos de aproximadamente 0.1 para alguns conjuntos de dados ao utilizar regressões com k vizinhos mais próximos considerando as leituras de \acrshort{o3} e \acrshort{mp10}.

\begin{figure}[h]
    \centering
    \caption{Resultados dos 10 melhores modelos de calibração aplicados as leituras de \acrshort{mp10} do sensor OPC-N3}
    \includegraphics[width=\textwidth]{chapters/3-RESULTADOS CAMPO/Figuras/pm10-all-models-performance.png}
    \label{fig:data-pm10-all-models-performance}
    \fonte{Desenvolvido pelo autor (2023)}
\end{figure}

A Figura \ref{fig:data-pm10-all-models-comparison} mostra uma comparação do desempenho, em termos do valor médio de R2, de cada modelo de regressão aplicado para diferentes combinações de variáveis de entrada.

\begin{figure}[h]
    \centering
    \caption{Comparação dos modelos de calibração aplicados as leituras de \acrshort{mp10} do sensor OPC-N3}
    \includegraphics[width=\textwidth]{chapters/3-RESULTADOS CAMPO/Figuras/pm10-all-models-comparison.png}
    \label{fig:data-pm10-all-models-comparison}
    \fonte{Desenvolvido pelo autor (2023)}
\end{figure}
