\begin{figure}[h!]
    \centering
    \caption{Resultados dos modelos de regressão aplicados às leituras dos sensores OX-B431}
    \includegraphics[width=\textwidth]{chapters/4-CALIBRAÇÃO MÚLTIPLOS SENSORES/Figuras/o3-b4-models-performance.png}
    \label{fig:data-o3-b4-models-performance}
\end{figure}

% ----------------------------------------------------------
\section{Correção das leituras dos sensores OX-B431 com as medições de referência}
% ----------------------------------------------------------

A partir dos dados de referência e das leituras de concentração e temperatura adquiridas pelo monitor em questão, foi realizada uma busca em grid para encontrar as melhores combinações de parâmetros e variáveis de entrada a modelos de regressão. As variáveis que foram testadas como entrada foram as leituras de concentração de \acrshort{o3} dos dois sensores OX-B431 e a temperatura no interior da câmara de medição. Como modelos de regressão foram testados: o Perceptron Multicamadas (MLP), a Regressão Linear Multivariada (MLR), os K Vizinhos mais Próximos (KNN) e as Florestas Aleatórias (RF). Na Tabela \ref{tab:data-o3-b4-calib-results} resumem-se os melhores modelos encontrados pela busca em \textit{grid} para calibrar as leituras dos sensores OX-B431. São mostradas as diferentes combinações de variáveis de entrada testadas em cada iteração da busca pelos melhores modelos de regressão. Os mesmos resultados são ilustrados graficamente na Figura \ref{fig:data-o3-b4-models-performance} que apresenta o desempenho dos modelos e as variáveis de entrada considerando os valores de r2, RMSE e MAE.

\begin{table}[h]
    \caption{Resultados da calibração dos sensores OX-B431}
    \centering
    \begin{tabularx}{0.95\textwidth}[h!]{
        >{\raggedright\hsize=.4\hsize\arraybackslash}X
        >{\raggedright\hsize=.6\hsize\arraybackslash}X 
        >{\raggedright\hsize=.6\hsize\arraybackslash}X
        >{\raggedright\hsize=.7\hsize\arraybackslash}X 
        >{\raggedright\hsize=.6\hsize\arraybackslash}X 
        >{\raggedright\hsize=.3\hsize\arraybackslash}X }
       \hline
       Var. & Modelo & R2 & RMSE & MAE & $\rho$\\ [0.5ex]
        \hline
        \acrshort{o3} (1) & \textbf{MLP}: & -0.38 ± 0.42 & -17.38 ± 3.49 & -13.72 ± 2.55 & 0.21 \\ [0.5ex]
           & \textbf{MLR} & -0.35 ± 0.38 & -17.26 ± 3.57 & -13.63 ± 2.58 & 0.22 \\ [0.5ex]
           & \textbf{KNN:} & -0.27 ± 0.31 & -17.24 ± 5.05 & -12.91 ± 3.92 & 0.21 \\ [0.5ex]
           & \textbf{RF:} & -0.68 ± 0.73 & -18.65 ± 3.27 & -14.87 ± 2.54 & 0.17 \\ [0.5ex]
        \hline
        \acrshort{o3} (1), T & \textbf{MLP:} & 0.42 ± 0.17 & -11.15 ± 0.98 & -8.57 ± 0.90 & 0.69 \\ [0.5ex]
            & \textbf{MLR:} & 0.36 ± 0.18 & -11.73 ± 1.23 & -9.12 ± 1.12 & 0.67 \\ [0.5ex]
            & \textbf{KNN:} & 0.21 ± 0.20 & -13.16 ± 1.93 & -9.96 ± 1.76 & 0.66 \\ [0.5ex]
            & \textbf{RF:} & 0.33 ± 0.19 & -11.99 ± 1.23 & -9.24 ± 1.12 & 0.62 \\ [0.5ex]
        \hline
        \acrshort{o3} (2) & \textbf{MLP}: & 0.16 ± 0.13 & -13.72 ± 2.11 & -10.87 ± 1.66 & 0.54 \\ [0.5ex]
           & \textbf{MLR} & 0.09 ± 0.14 & -14.34 ± 2.50 & -11.35 ± 2.13 & 0.56 \\ [0.5ex]
           & \textbf{KNN:} & 0.03 ± 0.35 & -14.27 +/- 1.49 & -11.27 ± 1.39 & 0.54 \\ [0.5ex]
           & \textbf{RF:} & -0.03 ± 0.37 & -14.76 ± 1.47 & -11.55 ± 1.34 & 0.52 \\ [0.5ex]
        \hline
        \acrshort{o3} (2), T & \textbf{MLP:} & 0.23 ± 0.17 & -13.16 ± 2.94 & -10.14 ± 2.62 & 0.67 \\ [0.5ex]
            & \textbf{MLR:} & 0.38 ± 0.22 & -11.44 ± 0.72 & -8.86 ± 0.86 & 0.69 \\ [0.5ex]
            & \textbf{KNN:} & 0.19 ± 0.34 & -12.91 ± 0.81 & -9.89 ± 0.92 & 0.70 \\ [0.5ex]
            & \textbf{RF:} & 0.28 ± 0.22 & -12.44 ± 1.11 & -9.54 ± 1.08 & 0.67 \\ [0.5ex]
        \hline
        \acrshort{o3} (1), \acrshort{o3} (2) & \textbf{MLP}: & 0.24 ± 0.15 & -13.01 ± 1.82 & -10.02 ± 1.55 & 0.60 \\ [0.5ex]
           & \textbf{MLR} & 0.14 ± 0.16 & -13.85 ± 2.26 & -10.81 ± 2.02 & 0.58 \\ [0.5ex]
           & \textbf{KNN:} & 0.12 ± 0.24 & -13.90 ± 1.86 & -10.86 ± 1.72 & 0.58 \\ [0.5ex]
           & \textbf{RF:} & 0.13 ± 0.24 & -13.70 ± 1.54 & -10.64 ± 1.43 & 0.57 \\ [0.5ex]
        \hline
        \acrshort{o3} (1), \acrshort{o3} (2), T & \textbf{MLP:} & 0.29 ± 0.21 & -12.37 ± 1.59 & -8.89 ± 0.82 & 0.71 \\ [0.5ex]
            & \textbf{MLR:} & 0.39 ± 0.21 & -11.37 ± 0.82 & -8.82 ± 0.88 & 0.69 \\ [0.5ex]
            & \textbf{KNN:} & 0.19 ± 0.34 & -12.97 ± 1.18 & -9.92 ± 1.37 & 0.72 \\ [0.5ex]
            & \textbf{RF:} & 0.30 ± 0.19 & -12.34 ± 1.61 & -9.57 ± 1.58 & 0.68 \\ [0.5ex]
        \hline
    \end{tabularx}
    \label{tab:data-o3-b4-calib-results}
\end{table}

\begin{figure}[h!]
    \centering
    \caption{Gráfico de dispersão das leituras dos sensores de \acrshort{o3} OX-B431 e a estação de referência após aplicar modelos de regressão considerando a temperatura}
    \begin{subfigure}{0.49\textwidth}
        \includegraphics[width=\textwidth]{chapters/4-CALIBRAÇÃO MÚLTIPLOS SENSORES/Figuras/o3-b4-1-T-MLP-Regression.png}
        \caption{Utilizando uma rede neural Perceptron Multicamadas obtiverem-se os melhores resultados de R2, RMSE e MAE considerando as leituras do sensor 1 e a temperatura}
        \label{fig:data-o3-1-T-reference-corr-MLP}
    \end{subfigure}
    \hfill
    \begin{subfigure}{0.49\textwidth}
        \includegraphics[width=\textwidth]{chapters/4-CALIBRAÇÃO MÚLTIPLOS SENSORES/Figuras/o3-b4-1-2-T-Multilinear-Regression.png}
        \caption{Utilizando uma regressão linear considerando as leituras dos sensores e a temperatura obtiveram-se resultados semelhantes}
        \label{fig:data-o3-1-2-T-reference-corr-MLR}
    \end{subfigure}
\end{figure}

De modo geral observa-se que os modelos que consideraram a temperatura como variável de entrada produziram os melhores resultados de r2, erro e correlação. Ao considerar os sensores de ozônio de forma independente, os resultados dos modelos não foram bons, com o sensor 2 gerando melhores resultados do que o sensor 1 (correlação e r2 maiores). Quando consideradas as leituras de ambos sensores, os modelos geraram melhores resultados, embora inferiores a quando considerada a temperatura. As Figuras \ref{fig:data-o3-1-T-reference-corr-MLP} e \ref{fig:data-o3-1-2-T-reference-corr-MLR} apresentam gráficos de dispersão com as leituras de referência e as inferidas pelos dois melhores modelos em termos de R2, RMSE e MAE.

% ----------------------------------------------------------
\section{Cálculo da concentração de Ozônio a partir das leituras do arranjo de sensores de gases}
% ----------------------------------------------------------

\begin{figure}[h!]
    \centering
    \caption{Desempenho dos modelos de regressão aplicados para inferir as leituras de concentração de \acrshort{o3} medidas pela estação de referência}
    \begin{subfigure}{0.9\textwidth}
        \includegraphics[width=\textwidth]{chapters/4-CALIBRAÇÃO MÚLTIPLOS SENSORES/Figuras/o3-all-models-performance.png}
        \caption{Valores de R2, RMSE e MAE obtidos pelos 10 modelos com maiores valores de R2}
        \label{fig:data-o3-all-models-performance}
    \end{subfigure}
    \begin{subfigure}{0.9\textwidth}
        \includegraphics[width=\textwidth]{chapters/4-CALIBRAÇÃO MÚLTIPLOS SENSORES/Figuras/o3-all-models-complexity.png}
        \caption{Modelos com menores valores de \acrshort{aic} e \acrshort{bic}}
        \label{fig:data-o3-all-models-comlexity}
    \end{subfigure}
    \label{fig:data-o3-all-models-performance-comlexity}
\end{figure}

A Figura \ref{fig:data-o3-all-models-performance} apresenta os valores de R2 dos 10 melhores modelos de calibração calculados para as leituras de \acrshort{co}. Observa-se que os valores de R2 desses 10 modelos oscilaram entre 0.2 e 0.6, todos incluíram o \acrshort{co} e a temperatura. Apenas dois modelos não foram regressões lineares, ocupando as posições 5 e 6 com regressões por Florestas Aleatórias. As Florestas, embora não tenham produzido os maiores valores de R2 médio, apresentaram menores valores de erro e valores de R2 máximo mais altos em comparação com os quatro modelos lineares que os antecederam. Já em termos de complexidade, observa-se que apenas o modelo de regressão por Florestas Aleatórias que considerou como variáveis de entrada as leituras de \acrshort{co}, dos dois sensores de \acrshort{o3}, de \acrshort{mp10} e temperatura, conseguiu estar entre os 10 com menores valores de \acrshort{aic}. Os modelos lineares por sua parte produziram os menores coeficientes de complexidade.

\begin{figure}[h]
    \centering
    \caption{Gráfico de dispersão das leituras do múltiplos sensores e a estação de referência para medição de \acrshort{o3}}
    \begin{subfigure}{0.49\textwidth}
        \includegraphics[width=\textwidth]{chapters/4-CALIBRAÇÃO MÚLTIPLOS SENSORES/Figuras/O3-co-no2-o31-pm10-T-Multilinear-Regression.png}
        \caption{Utilizando modelo de regressão linear multivariado com variáveis independentes: leituras de sensores CO-B4, NO2-B43F, OX-B431 (1), sensor de \acrshort{mp10} OPC-N3 e temperatura}
        \label{fig:data-co-no2-o31-pm10-T-reference-O3-corr-MLR}
    \end{subfigure}
    \hfill
    \begin{subfigure}{0.49\textwidth}
        \includegraphics[width=\textwidth]{chapters/4-CALIBRAÇÃO MÚLTIPLOS SENSORES/Figuras/O3-co-o31-o32-pm10-T-RF-Regression.png}
        \caption{Utilizando modelo de regressão de Florestas Aleatórias com variáveis independentes: leituras de sensores CO-B4, OX-B431 (1 e 2), sensor de \acrshort{mp10} OPC-N3 e temperatura}
        \label{fig:data-co-o31-o32-pm10-T-reference-O3-corr-RF}
    \end{subfigure}
\end{figure}

As Figuras \ref{fig:data-co-no2-o31-pm10-T-reference-O3-corr-MLR} e \ref{fig:data-co-o31-o32-pm10-T-reference-O3-corr-RF} mostram os resultados de aplicar modelos de calibração baseados numa regressão linear multivariada e em Florestas Aleatórias respectivamente. O primeiro considera como variáveis independentes as leituras dos sensores CO-B4, NO2-B43F, OX-B431 (1), sensor de \acrshort{mp10} OPC-N3 e temperatura. Já o segundo considera como variáveis de entrada as leituras dos sensores CO-B4, OX-B431 (1 e 2), sensor de \acrshort{mp10} OPC-N3 e temperatura.