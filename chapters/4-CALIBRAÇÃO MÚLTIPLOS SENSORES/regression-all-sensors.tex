
% ----------------------------------------------------------
\chapter{Metodologia de calibração utilizando as leituras de todos os sensores}\label{cap:calib-methodology}
% ----------------------------------------------------------

No Capítulo \ref{cap:field-monit-results} foram apresentados os resultados de aplicar modelos de regressão nas leituras dos sensores de gases para inferir a concentração real de cada poluente. Neles foram consideradas como variáveis de entrada de cada modelo a temperatura e as leituras do(s) sensor(es) especificados para cada composto gasoso. Assim, foram utilizados um sensor CO-B4 e a temperatura para medir \acrshort{co}; dois sensores OX-B431 e a temperatura para medir \acrshort{o3}; e um sensor NO2-B43F para medir \acrshort{no2}. No presente capítulo, serão aplicada uma metodologia de calibração similar ã do capítulo \ref{cap:field-monit-results}, mas dessa vez serão consideradas as leituras de todos os sensores para inferir a concentração de cada gás. Serão aplicadas buscas em grid para avaliar as combinações de variáveis de entrada e de parâmetros que produzam os melhores resultados de R2 e erro. Como a complexidade dos modelos aumenta com o incremento das variáveis independentes, também serão considerados as métricas de \acrshort{aic} e \acrshort{bic} na avaliação dos modelos de regressão.

% ----------------------------------------------------------
\section{Correção das leituras do sensor CO-B4 com as medições de referência}
% ----------------------------------------------------------

A partir dos dados de referência e das leituras de concentração e temperatura adquiridas pelo monitor em questão, foi realizada uma busca em grid para encontrar as melhores combinações de parâmetros e variáveis de entrada a modelos de regressão. As variáveis que foram testadas como entrada foram as leituras de concentração do sensor CO-B4 e a temperatura no interior da câmara de medição. Como modelos de regressão foram testados: o Perceptron Multicamadas (MLP), a Regressão Linear Multivariada (MLR), os K Vizinhos mais Próximos (KNN) e as Florestas Aleatórias (RF). Na Tabela \ref{tab:data-co-br-calib-results} resumem-se os melhores modelos encontrados pela busca em \textit{grid} para calibrar as leituras do sensor CO-B4. Os mesmos resultados são ilustrados graficamente na Figura \ref{fig:data-co-b4-models-performance} que apresenta o desempenho dos modelos e as variáveis de entrada considerando os valores de r2, RMSE e MAE.

\begin{table}[h!]
    \caption{Resultados da calibração do sensor CO-B4}
    \centering
    \begin{tabularx}{0.95\textwidth}[h!]{
         >{\raggedright\hsize=.2\hsize\arraybackslash}X
         >{\raggedright\hsize=.7\hsize\arraybackslash}X 
         >{\raggedright\hsize=.4\hsize\arraybackslash}X
         >{\raggedright\hsize=.4\hsize\arraybackslash}X 
         >{\raggedright\hsize=.3\hsize\arraybackslash}X 
         >{\raggedright\hsize=.3\hsize\arraybackslash}X }
        \hline
        Var. & Modelo & R2 & RMSE & MAE & $\rho$\\ [0.5ex]
        \hline
        CO & \textbf{MLP}: & -0.64 ± 0.49 & -0.07 ± 0.01 & -0.05 & 0.41 \\ [0.5ex]
           & \textbf{MLR} & -0.61 ± 0.47 & -0.07 ± 0.01 & -0.05 & 0.31 \\ [0.5ex]
           & \textbf{KNN:} & -0.47 ± 0.29 & -0.06 ± 0.01 & -0.05 & 0.37 \\ [0.5ex]
           & \textbf{RF:} & -0.60 ± 0.34 & -0.07 ± 0.01 & -0.05 & 0.30 \\ [0.5ex]
        \hline
        CO, T & \textbf{MLP:} & -0.59 ± 0.42 & -0.07 ± 0.01 & -0.05 & 0.46 \\ [0.5ex]
              & \textbf{MLR:} & -0.65 ± 0.45 & -0.07 ± 0.01 & -0.05 & 0.25 \\ [0.5ex]
              & \textbf{KNN:} & -0.68 ± 0.42 & -0.07 ± 0.01 & -0.05 & 0.51 \\ [0.5ex]
              & \textbf{RF:} & -0.70 ± 0.47 & -0.07 ± 0.01 & -0.05 & 0.48 \\ [0.5ex]
        \hline
    \end{tabularx}
    \label{tab:data-co-br-calib-results}
\end{table}

\begin{figure}[h]
    \centering
    \caption{Resultados dos modelos de regressão aplicados as leituras do sensor CO-B4}
    \includegraphics[width=\textwidth]{chapters/4-CALIBRAÇÃO MÚLTIPLOS SENSORES/Figuras/co-b4-models-performance.png}
    \label{fig:data-co-b4-models-performance}
\end{figure}

Como se observa todas as variantes de modelos e variáveis de entrada apresentaram valores de R2 negativos, indicando que nenhum dos modelos foi capaz de explicar a variância na variável dependente, i.e. a concentração real. Contudo, os modelos não lineares que incluíram a temperatura como variável de entrada, apresentaram melhorias na correlação entre a concentração real e a medida pelo sensor, com coeficientes de correlação de até 0.51. As Figuras \ref{fig:data-co-T-reference-corr-KNN} e \ref{fig:data-co-T-reference-corr-RF} apresentam os resultados ao aplicar os modelos de k Vizinhos mais Próximos e Florestas Aleatórias.

\begin{figure}[h]
    \centering
    \caption{Gráfico de dispersão das leituras do sensor CO-B4 e a estação de referência após aplicar modelos de regressão considerando a temperatura}
    \begin{subfigure}{0.495\textwidth}
        \includegraphics[width=\textwidth]{chapters/4-CALIBRAÇÃO MÚLTIPLOS SENSORES/Figuras/co-b4-T-KNN-Regression.png}
        \caption{Utilizando regressão pelos k Vizinhos mais Próximos}
        \label{fig:data-co-T-reference-corr-KNN}
    \end{subfigure}
    \hfill
    \begin{subfigure}{0.495\textwidth}
        \includegraphics[width=\textwidth]{chapters/4-CALIBRAÇÃO MÚLTIPLOS SENSORES/Figuras/co-b4-T-RF-Regression.png}
        \caption{Utilizando regressão pelas Florestas Aleatórias}
        \label{fig:data-co-T-reference-corr-RF}
    \end{subfigure}
\end{figure}

\begin{figure}[h!]
    \centering
    \caption{Desempenho dos modelos de regressão aplicados para inferir as leituras de concentração de \acrshort{co} medidas pela estação de referência}
    \begin{subfigure}{0.9\textwidth}
        \includegraphics[width=\textwidth]{chapters/4-CALIBRAÇÃO MÚLTIPLOS SENSORES/Figuras/co-all-models-performance.png}
        \caption{Valores de R2, RMSE e MAE obtidos pelos 10 modelos com maiores valores de R2}
        \label{fig:data-co-all-models-performance}
    \end{subfigure}
    \begin{subfigure}{0.9\textwidth}
        \includegraphics[width=\textwidth]{chapters/4-CALIBRAÇÃO MÚLTIPLOS SENSORES/Figuras/co-all-models-complexity.png}
        \caption{Modelos com menores valores de \acrshort{aic} e \acrshort{bic}}
        \label{fig:data-co-all-models-complexity}
    \end{subfigure}
    \label{fig:data-co-all-models-performance-comlexity}
\end{figure}

% ----------------------------------------------------------
\section{Cálculo da concentração de Monóxido de Carbono a partir das leituras do arranjo de sensores de gases}
% ----------------------------------------------------------

A Figura \ref{fig:data-co-all-models-performance} apresenta os valores de R2 dos 10 melhores modelos de calibração calculados para as leituras de \acrshort{co}. Observa-se que apesar do valor médio de R2 obtido nas validações cruzadas continuar sendo negativo, obtiveram-se máximos de até aproximadamente 0.1 para alguns conjuntos de dados de teste nas validações cruzadas. Em geral os modelos que produziram os melhores resultados foram baseados em regressões pelos k vizinhos mais próximos e redes neurais Perceptron Multicamadas. Com relação as variáveis de entrada dos modelos, todos os que produziram melhores resultados consideraram as leituras do sensor 1 de \acrshort{o3}. Outros que obtiveram valores positivos de R2 tiveram como variáveis de entrada o \acrshort{mp10} e \acrshort{no2}.

Ao comparar os modelos em termos de sua complexidade, observa-se uma sobreposição entre os que desempenharam melhor em termos de representação dos dados originais (maiores valores de R2) e os que desempenharam melhor em termos de complexidade (menores valores de AIC e BIC). A Figura \ref{fig:data-co-all-models-complexity} compara os valores de \acrshort{aic}, \acrshort{bic} e \acrshort{mse} dos 10 modelos que obtiveram menores valores de AIC. Por último as Figuras \ref{fig:data-co-o31-reference-corr-MLP} e \ref{fig:data-co-o31-mp10-reference-corr-KNN} mostram gráficos de dispersão entre os valores de saída dos modelos de calibração e os dados de referência de \acrshort{co}. Os gráficos mostram os resultados dos modelos com melhores valores de R2.

\begin{figure}[h]
    \centering
    \caption{Gráfico de dispersão das leituras do múltiplos sensores e a estação de referência para medição de \acrshort{co}}
    \begin{subfigure}{0.495\textwidth}
        \includegraphics[width=\textwidth]{chapters/4-CALIBRAÇÃO MÚLTIPLOS SENSORES/Figuras/CO-co-o31-mlp-Regression.png}
        \caption{Utilizando modelo de regressão com uma rede neural Perceptron Multicamadas; variáveis independentes: leituras de sensores CO-B4 e OX-B431 (1)}
        \label{fig:data-co-o31-reference-corr-MLP}
    \end{subfigure}
    \hfill
    \begin{subfigure}{0.495\textwidth}
        \includegraphics[width=\textwidth]{chapters/4-CALIBRAÇÃO MÚLTIPLOS SENSORES/Figuras/CO-co-o31-pm10-knn-Regression.png}
        \caption{Utilizando modelo de regressão pelos k vizinhos mais próximos; variáveis independentes: leituras de sensores CO-B4, OX-B431 (1) e de \acrshort{mp10} medido pelo OPC-N3}
        \label{fig:data-co-o31-mp10-reference-corr-KNN}
    \end{subfigure}
\end{figure}

\begin{figure}[h!]
    \centering
    \caption{Resultados dos modelos de regressão aplicados às leituras dos sensores OX-B431}
    \includegraphics[width=\textwidth]{chapters/4-CALIBRAÇÃO MÚLTIPLOS SENSORES/Figuras/o3-b4-models-performance.png}
    \label{fig:data-o3-b4-models-performance}
\end{figure}

% ----------------------------------------------------------
\section{Correção das leituras dos sensores OX-B431 com as medições de referência}
% ----------------------------------------------------------

A partir dos dados de referência e das leituras de concentração e temperatura adquiridas pelo monitor em questão, foi realizada uma busca em grid para encontrar as melhores combinações de parâmetros e variáveis de entrada a modelos de regressão. As variáveis que foram testadas como entrada foram as leituras de concentração de \acrshort{o3} dos dois sensores OX-B431 e a temperatura no interior da câmara de medição. Como modelos de regressão foram testados: o Perceptron Multicamadas (MLP), a Regressão Linear Multivariada (MLR), os K Vizinhos mais Próximos (KNN) e as Florestas Aleatórias (RF). Na Tabela \ref{tab:data-o3-b4-calib-results} resumem-se os melhores modelos encontrados pela busca em \textit{grid} para calibrar as leituras dos sensores OX-B431. São mostradas as diferentes combinações de variáveis de entrada testadas em cada iteração da busca pelos melhores modelos de regressão. Os mesmos resultados são ilustrados graficamente na Figura \ref{fig:data-o3-b4-models-performance} que apresenta o desempenho dos modelos e as variáveis de entrada considerando os valores de r2, RMSE e MAE.

\begin{table}[h]
    \caption{Resultados da calibração dos sensores OX-B431}
    \centering
    \begin{tabularx}{0.95\textwidth}[h!]{
        >{\raggedright\hsize=.4\hsize\arraybackslash}X
        >{\raggedright\hsize=.6\hsize\arraybackslash}X 
        >{\raggedright\hsize=.6\hsize\arraybackslash}X
        >{\raggedright\hsize=.7\hsize\arraybackslash}X 
        >{\raggedright\hsize=.6\hsize\arraybackslash}X 
        >{\raggedright\hsize=.3\hsize\arraybackslash}X }
       \hline
       Var. & Modelo & R2 & RMSE & MAE & $\rho$\\ [0.5ex]
        \hline
        \acrshort{o3} (1) & \textbf{MLP}: & -0.38 ± 0.42 & -17.38 ± 3.49 & -13.72 ± 2.55 & 0.21 \\ [0.5ex]
           & \textbf{MLR} & -0.35 ± 0.38 & -17.26 ± 3.57 & -13.63 ± 2.58 & 0.22 \\ [0.5ex]
           & \textbf{KNN:} & -0.27 ± 0.31 & -17.24 ± 5.05 & -12.91 ± 3.92 & 0.21 \\ [0.5ex]
           & \textbf{RF:} & -0.68 ± 0.73 & -18.65 ± 3.27 & -14.87 ± 2.54 & 0.17 \\ [0.5ex]
        \hline
        \acrshort{o3} (1), T & \textbf{MLP:} & 0.42 ± 0.17 & -11.15 ± 0.98 & -8.57 ± 0.90 & 0.69 \\ [0.5ex]
            & \textbf{MLR:} & 0.36 ± 0.18 & -11.73 ± 1.23 & -9.12 ± 1.12 & 0.67 \\ [0.5ex]
            & \textbf{KNN:} & 0.21 ± 0.20 & -13.16 ± 1.93 & -9.96 ± 1.76 & 0.66 \\ [0.5ex]
            & \textbf{RF:} & 0.33 ± 0.19 & -11.99 ± 1.23 & -9.24 ± 1.12 & 0.62 \\ [0.5ex]
        \hline
        \acrshort{o3} (2) & \textbf{MLP}: & 0.16 ± 0.13 & -13.72 ± 2.11 & -10.87 ± 1.66 & 0.54 \\ [0.5ex]
           & \textbf{MLR} & 0.09 ± 0.14 & -14.34 ± 2.50 & -11.35 ± 2.13 & 0.56 \\ [0.5ex]
           & \textbf{KNN:} & 0.03 ± 0.35 & -14.27 +/- 1.49 & -11.27 ± 1.39 & 0.54 \\ [0.5ex]
           & \textbf{RF:} & -0.03 ± 0.37 & -14.76 ± 1.47 & -11.55 ± 1.34 & 0.52 \\ [0.5ex]
        \hline
        \acrshort{o3} (2), T & \textbf{MLP:} & 0.23 ± 0.17 & -13.16 ± 2.94 & -10.14 ± 2.62 & 0.67 \\ [0.5ex]
            & \textbf{MLR:} & 0.38 ± 0.22 & -11.44 ± 0.72 & -8.86 ± 0.86 & 0.69 \\ [0.5ex]
            & \textbf{KNN:} & 0.19 ± 0.34 & -12.91 ± 0.81 & -9.89 ± 0.92 & 0.70 \\ [0.5ex]
            & \textbf{RF:} & 0.28 ± 0.22 & -12.44 ± 1.11 & -9.54 ± 1.08 & 0.67 \\ [0.5ex]
        \hline
        \acrshort{o3} (1), \acrshort{o3} (2) & \textbf{MLP}: & 0.24 ± 0.15 & -13.01 ± 1.82 & -10.02 ± 1.55 & 0.60 \\ [0.5ex]
           & \textbf{MLR} & 0.14 ± 0.16 & -13.85 ± 2.26 & -10.81 ± 2.02 & 0.58 \\ [0.5ex]
           & \textbf{KNN:} & 0.12 ± 0.24 & -13.90 ± 1.86 & -10.86 ± 1.72 & 0.58 \\ [0.5ex]
           & \textbf{RF:} & 0.13 ± 0.24 & -13.70 ± 1.54 & -10.64 ± 1.43 & 0.57 \\ [0.5ex]
        \hline
        \acrshort{o3} (1), \acrshort{o3} (2), T & \textbf{MLP:} & 0.29 ± 0.21 & -12.37 ± 1.59 & -8.89 ± 0.82 & 0.71 \\ [0.5ex]
            & \textbf{MLR:} & 0.39 ± 0.21 & -11.37 ± 0.82 & -8.82 ± 0.88 & 0.69 \\ [0.5ex]
            & \textbf{KNN:} & 0.19 ± 0.34 & -12.97 ± 1.18 & -9.92 ± 1.37 & 0.72 \\ [0.5ex]
            & \textbf{RF:} & 0.30 ± 0.19 & -12.34 ± 1.61 & -9.57 ± 1.58 & 0.68 \\ [0.5ex]
        \hline
    \end{tabularx}
    \label{tab:data-o3-b4-calib-results}
\end{table}

\begin{figure}[h!]
    \centering
    \caption{Gráfico de dispersão das leituras dos sensores de \acrshort{o3} OX-B431 e a estação de referência após aplicar modelos de regressão considerando a temperatura}
    \begin{subfigure}{0.49\textwidth}
        \includegraphics[width=\textwidth]{chapters/4-CALIBRAÇÃO MÚLTIPLOS SENSORES/Figuras/o3-b4-1-T-MLP-Regression.png}
        \caption{Utilizando uma rede neural Perceptron Multicamadas obtiverem-se os melhores resultados de R2, RMSE e MAE considerando as leituras do sensor 1 e a temperatura}
        \label{fig:data-o3-1-T-reference-corr-MLP}
    \end{subfigure}
    \hfill
    \begin{subfigure}{0.49\textwidth}
        \includegraphics[width=\textwidth]{chapters/4-CALIBRAÇÃO MÚLTIPLOS SENSORES/Figuras/o3-b4-1-2-T-Multilinear-Regression.png}
        \caption{Utilizando uma regressão linear considerando as leituras dos sensores e a temperatura obtiveram-se resultados semelhantes}
        \label{fig:data-o3-1-2-T-reference-corr-MLR}
    \end{subfigure}
\end{figure}

De modo geral observa-se que os modelos que consideraram a temperatura como variável de entrada produziram os melhores resultados de r2, erro e correlação. Ao considerar os sensores de ozônio de forma independente, os resultados dos modelos não foram bons, com o sensor 2 gerando melhores resultados do que o sensor 1 (correlação e r2 maiores). Quando consideradas as leituras de ambos sensores, os modelos geraram melhores resultados, embora inferiores a quando considerada a temperatura. As Figuras \ref{fig:data-o3-1-T-reference-corr-MLP} e \ref{fig:data-o3-1-2-T-reference-corr-MLR} apresentam gráficos de dispersão com as leituras de referência e as inferidas pelos dois melhores modelos em termos de R2, RMSE e MAE.

% ----------------------------------------------------------
\section{Cálculo da concentração de Ozônio a partir das leituras do arranjo de sensores de gases}
% ----------------------------------------------------------

\begin{figure}[h!]
    \centering
    \caption{Desempenho dos modelos de regressão aplicados para inferir as leituras de concentração de \acrshort{o3} medidas pela estação de referência}
    \begin{subfigure}{0.9\textwidth}
        \includegraphics[width=\textwidth]{chapters/4-CALIBRAÇÃO MÚLTIPLOS SENSORES/Figuras/o3-all-models-performance.png}
        \caption{Valores de R2, RMSE e MAE obtidos pelos 10 modelos com maiores valores de R2}
        \label{fig:data-o3-all-models-performance}
    \end{subfigure}
    \begin{subfigure}{0.9\textwidth}
        \includegraphics[width=\textwidth]{chapters/4-CALIBRAÇÃO MÚLTIPLOS SENSORES/Figuras/o3-all-models-complexity.png}
        \caption{Modelos com menores valores de \acrshort{aic} e \acrshort{bic}}
        \label{fig:data-o3-all-models-comlexity}
    \end{subfigure}
    \label{fig:data-o3-all-models-performance-comlexity}
\end{figure}

A Figura \ref{fig:data-o3-all-models-performance} apresenta os valores de R2 dos 10 melhores modelos de calibração calculados para as leituras de \acrshort{co}. Observa-se que os valores de R2 desses 10 modelos oscilaram entre 0.2 e 0.6, todos incluíram o \acrshort{co} e a temperatura. Apenas dois modelos não foram regressões lineares, ocupando as posições 5 e 6 com regressões por Florestas Aleatórias. As Florestas, embora não tenham produzido os maiores valores de R2 médio, apresentaram menores valores de erro e valores de R2 máximo mais altos em comparação com os quatro modelos lineares que os antecederam. Já em termos de complexidade, observa-se que apenas o modelo de regressão por Florestas Aleatórias que considerou como variáveis de entrada as leituras de \acrshort{co}, dos dois sensores de \acrshort{o3}, de \acrshort{mp10} e temperatura, conseguiu estar entre os 10 com menores valores de \acrshort{aic}. Os modelos lineares por sua parte produziram os menores coeficientes de complexidade.

\begin{figure}[h]
    \centering
    \caption{Gráfico de dispersão das leituras do múltiplos sensores e a estação de referência para medição de \acrshort{o3}}
    \begin{subfigure}{0.49\textwidth}
        \includegraphics[width=\textwidth]{chapters/4-CALIBRAÇÃO MÚLTIPLOS SENSORES/Figuras/O3-co-no2-o31-pm10-T-Multilinear-Regression.png}
        \caption{Utilizando modelo de regressão linear multivariado com variáveis independentes: leituras de sensores CO-B4, NO2-B43F, OX-B431 (1), sensor de \acrshort{mp10} OPC-N3 e temperatura}
        \label{fig:data-co-no2-o31-pm10-T-reference-O3-corr-MLR}
    \end{subfigure}
    \hfill
    \begin{subfigure}{0.49\textwidth}
        \includegraphics[width=\textwidth]{chapters/4-CALIBRAÇÃO MÚLTIPLOS SENSORES/Figuras/O3-co-o31-o32-pm10-T-RF-Regression.png}
        \caption{Utilizando modelo de regressão de Florestas Aleatórias com variáveis independentes: leituras de sensores CO-B4, OX-B431 (1 e 2), sensor de \acrshort{mp10} OPC-N3 e temperatura}
        \label{fig:data-co-o31-o32-pm10-T-reference-O3-corr-RF}
    \end{subfigure}
\end{figure}

As Figuras \ref{fig:data-co-no2-o31-pm10-T-reference-O3-corr-MLR} e \ref{fig:data-co-o31-o32-pm10-T-reference-O3-corr-RF} mostram os resultados de aplicar modelos de calibração baseados numa regressão linear multivariada e em Florestas Aleatórias respectivamente. O primeiro considera como variáveis independentes as leituras dos sensores CO-B4, NO2-B43F, OX-B431 (1), sensor de \acrshort{mp10} OPC-N3 e temperatura. Já o segundo considera como variáveis de entrada as leituras dos sensores CO-B4, OX-B431 (1 e 2), sensor de \acrshort{mp10} OPC-N3 e temperatura.

\begin{table}[h!]
    \caption{Resultados da calibração das leituras de \acrshort{no2} do sensor NO2-B43F}
    \centering
    \begin{tabularx}{0.95\textwidth}[h!]{
        >{\raggedright\hsize=.4\hsize\arraybackslash}X
        >{\raggedright\hsize=.6\hsize\arraybackslash}X 
        >{\raggedright\hsize=.6\hsize\arraybackslash}X
        >{\raggedright\hsize=.7\hsize\arraybackslash}X 
        >{\raggedright\hsize=.6\hsize\arraybackslash}X 
        >{\raggedright\hsize=.3\hsize\arraybackslash}X }
        \hline
        Var. & Modelo & R2 & RMSE & MAE & $\rho$\\ [0.5ex]
        \hline
        \acrshort{no2} & \textbf{MLP}: & -3.95 ± 5.68 & -12.87 ± 3.90 & -9.94 ± 2.52 & -- \\ [0.5ex]
           & \textbf{MLR} & -3.80 ± 6.39 & -12.61 ± 3.74 & -9.62 ± 2.17 & -- \\ [0.5ex]
           & \textbf{KNN:} & -3.73 ± 5.49 & -12.84 ± 3.67 & -9.92 ± 2.10 & -- \\ [0.5ex]
           & \textbf{RF:} & -4.39 ± 6.07 & -13.70 ± 3.50 & -10.33 ± 2.08 & -- \\ [0.5ex]
        \hline
        \acrshort{no2}, T & \textbf{MLP:} & -2.84 ± 4.36 & -11.76 ± 3.40 & -9.00 ± 1.81 & 0.59 \\ [0.5ex]
              & \textbf{MLR:} & -3.62 ± 5.88 & -12.57 ± 3.40 & -9.70 ± 1.87 & -- \\ [0.5ex]
              & \textbf{KNN:} & -2.62 ± 4.52 & -12.02 ± 3.33 & -8.88 ± 1.54 & 0.61 \\ [0.5ex]
              & \textbf{RF:} & -2.66 ± 4.51 & -11.93 ± 3.39 & -9.18 ± 2.03 & 0.63 \\ [0.5ex]
        \hline
    \end{tabularx}
    \label{tab:data-no2-calib-results}
\end{table}

% ----------------------------------------------------------
\section{Correção das leituras do sensor NO2-B43F com as medições de referência}
% ----------------------------------------------------------

A partir dos dados de referência e das leituras de concentração e temperatura adquiridas pelo monitor em questão, foi realizada uma busca em grid para encontrar as melhores combinações de parâmetros e variáveis de entrada a modelos de regressão. As variáveis que foram testadas como entrada foram as leituras de concentração de \acrshort{no2} do sensor NO2-B43F e a temperatura no interior da câmara de medição. Na Tabela \ref{tab:data-no2-calib-results} resumem-se os melhores modelos encontrados pela busca em \textit{grid} para calibrar as leituras do sensor de baixo custo. Os mesmos resultados são ilustrados graficamente na Figura \ref{fig:data-no2-models-performance} que apresenta o desempenho dos modelos e as variáveis de entrada considerando os valores de R2, RMSE e MAE.

\begin{figure}[h!]
    \centering
    \caption{Resultados dos modelos de calibração aplicados as leituras de \acrshort{no2} do sensor NO2-B43F}
    \includegraphics[width=0.95\textwidth]{chapters/4-CALIBRAÇÃO MÚLTIPLOS SENSORES/Figuras/no2-B43F-models-performance.png}
    \label{fig:data-no2-models-performance}
\end{figure}

\begin{figure}[h!]
    \centering
    \caption{Gráfico de dispersão das leituras do sensor de \acrshort{no2} NO2-B43F e a estação de referência após aplicar modelos de regressão considerando a temperatura}
    \begin{subfigure}{0.49\textwidth}
        \includegraphics[width=\textwidth]{chapters/4-CALIBRAÇÃO MÚLTIPLOS SENSORES/Figuras/no2-T-KNN-Regression.png}
        \caption{Utilizando uma regressão pelos k vizinhos mais próximos considerando a temperatura obteve-se um $\rho$ de 0.61}
        \label{fig:data-no2-T-reference-corr-KNN}
    \end{subfigure}
    \hfill
    \begin{subfigure}{0.49\textwidth}
        \includegraphics[width=\textwidth]{chapters/4-CALIBRAÇÃO MÚLTIPLOS SENSORES/Figuras/no2-T-RF-Regression.png}
        \caption{Utilizando uma regressão por Florestas Aleatórias considerando a temperatura obteve-se um valor de $\rho$ de 0.63}
        \label{fig:data-no2-T-reference-corr-RF}
    \end{subfigure}
\end{figure}

Como se observa todas as variantes de modelos e variáveis de entrada apresentaram valores de R2 negativos, indicando que nenhum dos modelos foi capaz de explicar a variância na variável dependente, i.e. a concentração real. Observa-se também que nenhum dos modelos uni-variados conseguiu que os dados observados e os inferidos apresentassem alguma correlação. Por outro lado, os modelos multivariados, a exceção da regressão linear, que incluíram a temperatura como variável de entrada, produziram coeficientes de correlação entre a concentração real e a medida pelo sensor, entre 0.59 - 0.63. As Figuras \ref{fig:data-no2-T-reference-corr-KNN} e \ref{fig:data-no2-T-reference-corr-RF} apresentam os resultados ao aplicar os modelos de k Vizinhos mais Próximos e Florestas Aleatórias.

\begin{figure}[h!]
    \centering
    \caption{Desempenho dos modelos de regressão aplicados para inferir as leituras de concentração de \acrshort{no2} medidas pela estação de referência}
    \begin{subfigure}{0.9\textwidth}
        \includegraphics[width=\textwidth]{chapters/4-CALIBRAÇÃO MÚLTIPLOS SENSORES/Figuras/no2-all-models-performance.png}
        \caption{Valores de R2, RMSE e MAE obtidos pelos 10 modelos com maiores valores de R2}
        \label{fig:data-no2-all-models-performance}
    \end{subfigure}
    \begin{subfigure}{0.9\textwidth}
        \includegraphics[width=\textwidth]{chapters/4-CALIBRAÇÃO MÚLTIPLOS SENSORES/Figuras/no2-all-models-complexity.png}
        \caption{Modelos com menores valores de \acrshort{aic} e \acrshort{bic}}
        \label{fig:data-no2-all-models-comlexity}
    \end{subfigure}
    \label{fig:data-no2-all-models-performance-comlexity}
\end{figure}

% ----------------------------------------------------------
\section{Cálculo da concentração de Dióxido de Nitrogênio a partir das leituras do arranjo de sensores de gases}
% ----------------------------------------------------------

A Figura \ref{fig:data-no2-all-models-performance} apresenta os valores de R2 dos 10 melhores modelos de calibração calculados para as leituras de \acrshort{no2}. Observa-se que os valores de R2 desses 10 modelos apresentaram valores de R2 em média positivos, com valores máximos de até aproximadamente 0.2, todos obtidos a partir de regressões lineares. Com relação as variáveis de entrada observa-se que todos os 10 modelos consideraram o \acrshort{co} com variações nos restantes das variáveis para cada modelo. Com relação à complexidade dos modelos (Figura \ref{fig:data-no2-all-models-performance-comlexity}) observa-se que o ranqueamento por \acrshort{aic} coincidiu bastante com o ranqueamento por R2. As Figuras \ref{fig:data-co-no2-reference-NO2-corr-MLR} e \ref{fig:data-co-no2-pm10-reference-NO2-corr-RF} mostram os resultados obtidos com os dois modelos com maior R2 médio, i.e.: regressões lineares com variáveis de entrada leituras de sensores CO-B4 e NO2B43F, e leituras de sensores CO-B4, NO2B43F e sensor de \acrshort{mp10} OPC-N3, respectivamente. As figuras mostram gráficos de dispersão entre os dados calibrados por esses modelos e as leituras de referência.

\begin{figure}[h]
    \centering
    \caption{Gráfico de dispersão das leituras de múltiplos sensores e a estação de referência para medição de \acrshort{no2}}
    \begin{subfigure}{0.49\textwidth}
        \includegraphics[width=\textwidth]{chapters/4-CALIBRAÇÃO MÚLTIPLOS SENSORES/Figuras/NO2-co-no2-Multilinear-Regression.png}
        \caption{Utilizando modelo de regressão linear multivariado com variáveis independentes: leituras de sensores CO-B4, e NO2-B43F}
        \label{fig:data-co-no2-reference-NO2-corr-MLR}
    \end{subfigure}
    \hfill
    \begin{subfigure}{0.49\textwidth}
        \includegraphics[width=\textwidth]{chapters/4-CALIBRAÇÃO MÚLTIPLOS SENSORES/Figuras/NO2-co-no2-pm10-Multilinear-Regression.png}
        \caption{Utilizando modelo de regressão linear multivariado com variáveis independentes: leituras de sensores CO-B4, NO2-B43F e sensor de \acrshort{mp10} OPC-N3}
        \label{fig:data-co-no2-pm10-reference-NO2-corr-RF}
    \end{subfigure}
\end{figure}

\begin{figure}[h!]
    \centering
    \caption{Desempenho dos modelos de regressão aplicados para inferir as leituras de concentração de \acrshort{mp10} medidas pela estação de referência}
    \begin{subfigure}{0.9\textwidth}
        \includegraphics[width=\textwidth]{chapters/4-CALIBRAÇÃO MÚLTIPLOS SENSORES/Figuras/pm10-all-models-performance.png}
        \caption{Valores de R2, RMSE e MAE obtidos pelos 10 modelos com maiores valores de R2}
        \label{fig:data-pm10-all-models-performance}
    \end{subfigure}
    \begin{subfigure}{0.9\textwidth}
        \includegraphics[width=\textwidth]{chapters/4-CALIBRAÇÃO MÚLTIPLOS SENSORES/Figuras/pm10-all-models-complexity.png}
        \caption{Modelos com menores valores de \acrshort{aic} e \acrshort{bic}}
        \label{fig:data-pm10-all-models-comlexity}
    \end{subfigure}
    \label{fig:data-pm10-all-models-performance-comlexity}
\end{figure}

% ----------------------------------------------------------
\section{Calibração das leituras de Material Particulado }
% ----------------------------------------------------------

A Figura \ref{fig:data-pm10-all-models-performance} apresenta os valores de R2 dos 10 melhores modelos de calibração calculados para as leituras de \acrshort{mp10}. Observa-se que os valores de R2 desses 10 modelos apresentaram valores de R2 em média positivos, com valores mínimos e máximos oscilando entre -0.1 e 0.2, todos obtidos a partir de regressões lineares. Com relação as variáveis de entrada observa-se maior variabilidade do que nos casos analisados anteriormente, mas em geral destaca-se a presença de sensores de \acrshort{o3} em 9 dos modelos e em segundo lugar da temperatura, presente em 6. Nenhum dos modelos incluiu leituras do sensor NO2-B43F. Com relação à complexidade dos modelos (Figura \ref{fig:data-pm10-all-models-performance-comlexity}) observa-se bastante coincidência com ranqueamento de R2. As Figuras \ref{fig:data-co-o31-o31-pm10-reference-PM10-corr-MLR} e \ref{fig:data-o31-o32-pm10-reference-PM10-corr-RF} mostram os resultados obtidos com os dois modelos com maior R2 médio, i.e.: regressões lineares com variáveis de entrada leituras de sensores CO-B4, OX-B431 (1 e 2) e sensor de \acrshort{mp10} OPC-N3, e leituras de sensores OX-B431 (1 e 2) e sensor de \acrshort{mp10} OPC-N3, respectivamente. As figuras mostram gráficos de dispersão entre os dados calibrados por esses modelos e as leituras de referência.

\begin{figure}[h]
    \centering
    \caption{Gráfico de dispersão das leituras do múltiplos sensores e a estação de referência para medição de \acrshort{mp10}}
    \begin{subfigure}{0.49\textwidth}
        \includegraphics[width=\textwidth]{chapters/4-CALIBRAÇÃO MÚLTIPLOS SENSORES/Figuras/MP10-co-o31-o32-pm10-Multilinear-Regression.png}
        \caption{Utilizando modelo de regressão linear multivariado com variáveis independentes: leituras de sensores CO-B4, OX-B431 (1 e 2) e sensor de \acrshort{mp10} OPC-N3}
        \label{fig:data-co-o31-o31-pm10-reference-PM10-corr-MLR}
    \end{subfigure}
    \hfill
    \begin{subfigure}{0.49\textwidth}
        \includegraphics[width=\textwidth]{chapters/4-CALIBRAÇÃO MÚLTIPLOS SENSORES/Figuras/MP10-o31-o32-pm10-Multilinear-Regression.png}
        \caption{Utilizando modelo de regressão linear multivariado com variáveis independentes: leituras de sensores OX-B431 (1 e 2) e sensor de \acrshort{mp10} OPC-N3}
        \label{fig:data-o31-o32-pm10-reference-PM10-corr-RF}
    \end{subfigure}
\end{figure}


\section{Discussão}

Na Tabela \ref{tab:summary-calib-results-all-sensors} resumem-se os resultados dos melhores modelos para cada poluente. Observa-se que de modo geral os parâmetros utilizados para medir o desempenho dos modelos melhorou com relação aos mostrados no Capítulo \ref{cap:field-monit-results}. Os valores de R2, embora ainda pequenos, passaram a ser positivos com exceção do \acrshort{co}, que no entanto mostrou valores de R2 máximo de aproximadamente 0.1. Os valores de erro quadrático e erro absoluto se mantiveram na mesma faixa de valores. Os modelos escolhidos apresentaram menores valores de coeficientes de correlação de Spearman devido a que o parâmetro escolhido para a otimização dos modelos foi o coeficiente de determinação R2.

\begin{table}[h!]
    \caption{Resultados dos melhores modelos por poluente considerando as leituras de todos os sensores nos modelos}
    \centering
    \begin{tabularx}{0.95\textwidth}[h!]{
         >{\raggedright\hsize=.4\hsize\arraybackslash}X
         >{\raggedright\hsize=.2\hsize\arraybackslash}X 
         >{\raggedright\hsize=.4\hsize\arraybackslash}X
         >{\raggedright\hsize=.4\hsize\arraybackslash}X 
         >{\raggedright\hsize=.4\hsize\arraybackslash}X 
         >{\raggedright\hsize=.3\hsize\arraybackslash}X }
        \hline
        Poluente & Modelo & R2 & RMSE & MAE & $\rho$\\ [0.5ex]
        \hline
        \acrshort{co} & MLP & -0.25 ± 0.28 & -0.06 ± 0.01 & -0.05 & 0.44 \\ [0.5ex]
        \acrshort{o3} & MLR & 0.42 ± 0.11 & -11.00 ± 1.92 & -8.67 ± 1.55 & 0.72 \\ [0.5ex]
        \acrshort{no2} & MLR & 0.10 ± 0.04 & -8.43 ± 2.41 & -6.51 ± 1.92 & 0.27 \\ [0.5ex]
        \acrshort{mp10} & MLR & 0.07 ± 0.12 & -9.90 ± 1.69 & -7.41 ± 1.49 & 0.34 \\ [0.5ex]
        \hline
    \end{tabularx}
    \label{tab:summary-calib-results-all-sensors}
\end{table}

Ao comparar os resultados obtidos neste trabalho com os compilados por \cite{Kang2022PerformanceReview} observa-se que os resultados obtidos pelo instrumento desenvolvido são inferiores à maioria dos trabalhos reportados com sensores de baixo custo. Por exemplo, entre os resultados reportados para a medição de \acrshort{mp10}, nenhum obteve valores de R2 inferiores a 0.2, sendo que a mediana dos valores reportados em medições em área externa é de 0.71, e dentre esses, a mediana dos que utilizam sensores Alphasense é de 0.68.

Se considerados os resultados para \acrshort{co}, observa-se que valores de R2 inferiores a 0.1 representam menos de 10 \% do total de trabalhos realizados em área externa, nenhum deles utilizando sensores Alphasense. A mediana de R2 obtida em área externa é de 0.68. As medições de \acrshort{no2} encontram-se num patamar similar; menos de 10 \% dos trabalhos em área externa em total, e menos de 10 \% dos realizados em área externa com sensores Alphasense obtiveram valores de R2 de 0.1. As medianas de R2 para cada caso são de 0.57 e 0.68 respectivamente.

Como era de esperar os melhores resultados obtiveram-se nas medições de \acrshort{o3}, onde o valor de 0.42 de R2 obtido localiza o presente trabalho entre 20 e 25 \% do total de trabalhos em área externa e entre entre 20 e 25 \% do total de trabalhos em área externa com sensores Alphasense. As medianas para esses dois grupos foram de 0.78 e 0.42 respectivamente.

Apesar de que os resultados em comparação com outros trabalhos foram inferiores, ao considerar a evolução das leituras desde os dados brutos até os modelos multivariados considerando todos os sensores percebe-se uma melhoria considerável na qualidade das leituras do instrumento desenvolvido. Assim, a metodologia de calibração proposta apresenta-se como mais uma alternativa para calibração de outros monitores de baixo custo. Essa metodologia, contudo, tem a ressalva de aumentar a complexidade dos modelos de calibração.

Cabe destacar que a temperatura ganhou menor destaque nesses modelos de regressão, já que apenas os modelos de \acrshort{o3} incluíram a temperatura como variável independente. Por outro lado, todos os poluentes a exceção do \acrshort{no2} consideraram as leituras de pelo menos um dos sensores OX-B431. Dada a forte relação desses sensores com a temperatura, é possível que a sua inclusão tenha substituído a relação com a temperatura. A ausência de temperatura no modelo de \acrshort{no2} faz sentido dada a relação mostrada entre as variáveis.