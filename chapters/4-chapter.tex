% ----------------------------------------------------------
\chapter{Conclusão}
% ----------------------------------------------------------

Neste trabalho foi desenvolvida uma rede colaborativa para monitorização da qualidade do ar de baixo custo, a iniciativa CLEAN. Este iniciativa promove a colaboração para o desenvolvimento de plataformas de monitoramento de baixo custo, com custos mais baixos e com maior flexibilidade do que as atuais iniciativas de acesso aberto. A rede colaborativa pode incorporar um grupo mais amplo e diversificado de especialistas e entusiastas do monitoramento ambiental, aumentando a quantidade de dados disponíveis publicamente, diversificando as aplicações de monitoramento e melhorando a cobertura espaço-temporal da monitorização da qualidade do ar, especialmente nos países em desenvolvimento. CLEAN proporciona recursos de hardware e bibliotecas de firmware para diversas aplicações e usuários, assim como oferece uma aplicação backend e uma API REST, que facilita a integração de novos dispositivos à rede. Essa infraestrutura completa, aliada a uma aplicação frontend para visualização intuitiva dos dados, estabelece um ambiente para a integração de novos usuários e monitores, aumentando e diversificando o número de aplicações de monitoramento. O diferencial desta iniciativa com outras similares levantadas na revisão bibliográfica é que a estrutura modular das bibliotecas de firmware e a API para envio e leitura de dados para e do banco de dados contribuem na incorporação de novos colaboradores provém flexibilidade se para adaptar às especificações de cada aplicação e ao mesmo tempo uma estrutura sólida como base do desenvolvimento.

A iniciativa CLEAN promove a ciência cidadã, facilitando o processo de desenvolvimento dos dispositivos a um público mais vasto e disponibilizando dados sobre a qualidade do ar para análise e visualização. A rede de colaboradores pode se expandir para desenvolvedores, pesquisadores, amadores e estudantes que possam utilizar as ferramentas disponíveis no CLEAN para educação, prototipagem, uso pessoal e pesquisa. Uma cobertura mais ampla dos dados sobre a qualidade do ar disponibilizará uma quantidade considerável de dados, facilitando o acesso à informação ambiental para tornar as cidades e as povoações mais inclusivas, seguras, resilientes e sustentáveis. O acesso a informações representativas e fiáveis ajuda os investigadores e os decisores a encontrar soluções que promovam o florescimento de cidades mais inteligentes e saudáveis. Nos países em desenvolvimento, onde o acesso a tecnologias caras é mais limitado, iniciativas como o CLEAN são alternativas interessantes às redes de monitorização altamente dispendiosas que podem contribuir para o crescimento econômico sustentado e para um aumento da qualidade de vida dos cidadãos. Esta iniciativa tem potencial para expandir a comunidade de monitoramento do ar, especialmente no território brasileiro, e facilitar o acesso a informações sobre poluição atmosférica em regiões onde os dados regulatórios são escassos ou mesmo inexistentes.

Como parte da pesquisa também foram desenvolvidos e adicionados à rede, cinco dispositivos de medição de qualidade do ar, baseados no framework Arduino. Para validação das medições um dos equipamentos foi instalado junto a uma estação de referência por um período de 6 meses e suas leituras foram calibradas. Observou-se que as medições de \acrshort{no2} e \acrshort{so2} não puderam ser aproveitadas, indicando a necessidade de refinamento na aquisição de sinal para redução de ruído e estabilização do sinal de linha base. Os testes aprofundados com modelos de regressão multivariada, perceptron multicamadas, k vizinhos mais próximos e florestas aleatórias para calibrar as leituras de \acrshort{co}, \acrshort{o3} e \acrshort{mp10} revelam uma abordagem metódica e robusta em busca de precisão. A análise apurada indicou que as leituras de \acrshort{o3} produziram os melhores resultados com valores de r2 de 0.4. Observou-se também que quando combinadas com as leituras dos outros sensores aumentou a precisão das medições.

Como próximos passos, recomenda-se uma atenção especial para o aprimoramento do hardware de aquisição de sinal, com o objetivo de reduzir o ruído e estabilizar o sinal de linha base. Essas melhorias podem resultar em medições ainda mais precisas e confiáveis, consolidando a contribuição significativa deste trabalho na promoção do monitoramento de baixo custo da qualidade do ar em diversos contextos.