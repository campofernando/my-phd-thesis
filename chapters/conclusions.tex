% ----------------------------------------------------------
\chapter*{Conclusão}
% ----------------------------------------------------------
\phantomsection
\addcontentsline{toc}{chapter}{CONCLUSÃO}%
\markboth{Conclusão}{Conclusão}%

Neste trabalho foi desenvolvida uma rede colaborativa para monitorização da qualidade do ar de baixo custo, a iniciativa CLEAN. Este iniciativa promove a colaboração para o desenvolvimento de plataformas de monitoramento de baixo custo, com custos mais baixos e com maior flexibilidade do que as atuais iniciativas de acesso aberto. A rede colaborativa pode incorporar um grupo mais amplo e diversificado de especialistas e entusiastas do monitoramento ambiental, aumentando a quantidade de dados disponíveis publicamente, diversificando as aplicações de monitoramento e melhorando a cobertura espaço-temporal da monitorização da qualidade do ar, especialmente nos países em desenvolvimento. CLEAN proporciona recursos de hardware e bibliotecas de firmware para diversas aplicações e usuários, assim como oferece uma aplicação backend e uma API REST, que facilita a integração de novos dispositivos à rede. Essa estrutura completa, aliada a uma aplicação frontend para visualização intuitiva dos dados, estabelece um ambiente para a integração de novos usuários e monitores, aumentando e diversificando o número de aplicações de monitoramento. O diferencial desta iniciativa com outras similares levantadas na revisão bibliográfica é que a estrutura modular das bibliotecas de firmware e a API para envio e leitura de dados para e do banco de dados contribuem na incorporação de novos colaboradores provém flexibilidade se para adaptar às especificações de cada aplicação e ao mesmo tempo uma estrutura sólida como base do desenvolvimento.

A iniciativa CLEAN promove a ciência cidadã, facilitando o processo de desenvolvimento dos dispositivos a um público mais vasto e disponibilizando dados sobre a qualidade do ar para análise e visualização. A rede de colaboradores pode se expandir para desenvolvedores, pesquisadores, amadores e estudantes que possam utilizar as ferramentas disponíveis no CLEAN para educação, prototipagem, uso pessoal e pesquisa. Uma cobertura mais ampla dos dados sobre a qualidade do ar disponibilizará uma quantidade considerável de dados, facilitando o acesso à informação ambiental para tornar as cidades e as povoações mais inclusivas, seguras, resilientes e sustentáveis. O acesso a informações representativas e fiáveis ajuda os investigadores e os decisores a encontrar soluções que promovam o florescimento de cidades mais inteligentes e saudáveis. Nos países em desenvolvimento, onde o acesso a tecnologias caras é mais limitado, iniciativas como o CLEAN são alternativas interessantes às redes de monitorização altamente dispendiosas que podem contribuir para o crescimento econômico sustentado e para um aumento da qualidade de vida dos cidadãos. Esta iniciativa tem potencial para expandir a comunidade de monitoramento do ar, especialmente no território brasileiro, e facilitar o acesso a informações sobre poluição atmosférica em regiões onde os dados regulatórios são escassos ou mesmo inexistentes.

Como parte da pesquisa também foram desenvolvidos cinco dispositivos de medição de qualidade do ar, baseados no framework Arduino. Dois deles foram protótipos para medições em local fixo e medições móveis. Os outros três dispositivos foram desenvolvidos a partir de uma placa de circuito impresso versionada como CLEAN Arduino Mega possuindo melhor acabamento e robustez. Um desses equipamentos foi instalado por um período de 5 meses junto a uma estação de referência no município de Tubarão para validação e correção das suas leituras. Os outros dois equipamentos foram instalados posteriormente no mesmo local e se encontram coletando dados até o momento. Das leituras coletadas pelo primeiro dispositivo CLEAN Arduino Mega instalado em campo observou-se que as principais causas de dados inválidos foram as alterações de linha base e os períodos sem dados coletados devido a quedas de energia no local, falhas na comunicação com os sensores e falhas do equipamento. Versões futuras podem melhorar as interfaces de comunicação com os sensores utilizando placas dedicadas, com microcontrolador próprio que enviem as leituras de concentração por um canal digital e não mais analógico. Dessa forma isola-se a etapa de aquisição do restante do equipamento, adicionando maior proteção a ruído e interferências elétricas. Prover aos equipamentos de uma infraestrutura própria de energia e acesso à internet são outras ações que podem ser tomadas para acrescentar robustez e autonomia aos dispositivos, deixando-os menos propensos a falhas originadas na infraestrutura do local de instalação.

Em comparação com outros trabalhos que instalam equipamentos de baixo custo junto a estações certificadas em campo, o equipamento desenvolvido teve um baixo desempenho na medição dos poluentes. Contudo, com a metodologia proposta de correção de dados foi possível agregar valor às leituras. Corroborou-se que, considerando as leituras do conjunto de sensores é possível corrigir, ainda que com certo grau de erro, um conjunto de medições muito ruidosas e com dados faltantes, como foi o caso das medições de \acrshort{no2}. Além da metodologia de correção proposta, foi desenvolvida uma metodologia para o pré-tratamento dos dados utilizando várias técnicas reportadas na literatura. Essa metodologia possibilita detectar falhas e remover dados corrompidos, e pode ser incorporada aos equipamentos de medição para realizar validações dos dados em linha e em tempo real. Por exemplo, se uma batelada de medições coletadas pelo equipamento não tiver o 75 \% dos dados válidos ela poderia ser descartada ou marcada para posterior correção por um serviço de nuvem. Se um equipamento apresentar sucessivas bateladas com falha, um alerta poderia ser gerado indicando uma falha de funcionamento.

Como próximos passos, recomenda-se uma atenção especial para o aprimoramento do hardware de aquisição de sinal, com o objetivo de reduzir o ruído e estabilizar o sinal de linha base. Essas melhorias podem resultar em medições ainda mais precisas e confiáveis, consolidando a contribuição significativa deste trabalho na promoção do monitoramento de baixo custo da qualidade do ar em diversos contextos. Recomenda-se também adicionar uma rotina de validação dos dados de forma contínua com notificações do estado do equipamento para o serviço em nuvem e adicionar hardware aos equipamentos para funcionarem de forma independente da infraestrutura do local de instalação, utilizando conjunto de bateria e painel solar e explorando tecnologias de conectividade GPRS ou LoRa.